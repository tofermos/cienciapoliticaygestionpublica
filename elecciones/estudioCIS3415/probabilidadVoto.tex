% Options for packages loaded elsewhere
\PassOptionsToPackage{unicode}{hyperref}
\PassOptionsToPackage{hyphens}{url}
%
\documentclass[
]{article}
\usepackage{amsmath,amssymb}
\usepackage{iftex}
\ifPDFTeX
  \usepackage[T1]{fontenc}
  \usepackage[utf8]{inputenc}
  \usepackage{textcomp} % provide euro and other symbols
\else % if luatex or xetex
  \usepackage{unicode-math} % this also loads fontspec
  \defaultfontfeatures{Scale=MatchLowercase}
  \defaultfontfeatures[\rmfamily]{Ligatures=TeX,Scale=1}
\fi
\usepackage{lmodern}
\ifPDFTeX\else
  % xetex/luatex font selection
\fi
% Use upquote if available, for straight quotes in verbatim environments
\IfFileExists{upquote.sty}{\usepackage{upquote}}{}
\IfFileExists{microtype.sty}{% use microtype if available
  \usepackage[]{microtype}
  \UseMicrotypeSet[protrusion]{basicmath} % disable protrusion for tt fonts
}{}
\makeatletter
\@ifundefined{KOMAClassName}{% if non-KOMA class
  \IfFileExists{parskip.sty}{%
    \usepackage{parskip}
  }{% else
    \setlength{\parindent}{0pt}
    \setlength{\parskip}{6pt plus 2pt minus 1pt}}
}{% if KOMA class
  \KOMAoptions{parskip=half}}
\makeatother
\usepackage{xcolor}
\usepackage[margin=1in]{geometry}
\usepackage{color}
\usepackage{fancyvrb}
\newcommand{\VerbBar}{|}
\newcommand{\VERB}{\Verb[commandchars=\\\{\}]}
\DefineVerbatimEnvironment{Highlighting}{Verbatim}{commandchars=\\\{\}}
% Add ',fontsize=\small' for more characters per line
\usepackage{framed}
\definecolor{shadecolor}{RGB}{248,248,248}
\newenvironment{Shaded}{\begin{snugshade}}{\end{snugshade}}
\newcommand{\AlertTok}[1]{\textcolor[rgb]{0.94,0.16,0.16}{#1}}
\newcommand{\AnnotationTok}[1]{\textcolor[rgb]{0.56,0.35,0.01}{\textbf{\textit{#1}}}}
\newcommand{\AttributeTok}[1]{\textcolor[rgb]{0.13,0.29,0.53}{#1}}
\newcommand{\BaseNTok}[1]{\textcolor[rgb]{0.00,0.00,0.81}{#1}}
\newcommand{\BuiltInTok}[1]{#1}
\newcommand{\CharTok}[1]{\textcolor[rgb]{0.31,0.60,0.02}{#1}}
\newcommand{\CommentTok}[1]{\textcolor[rgb]{0.56,0.35,0.01}{\textit{#1}}}
\newcommand{\CommentVarTok}[1]{\textcolor[rgb]{0.56,0.35,0.01}{\textbf{\textit{#1}}}}
\newcommand{\ConstantTok}[1]{\textcolor[rgb]{0.56,0.35,0.01}{#1}}
\newcommand{\ControlFlowTok}[1]{\textcolor[rgb]{0.13,0.29,0.53}{\textbf{#1}}}
\newcommand{\DataTypeTok}[1]{\textcolor[rgb]{0.13,0.29,0.53}{#1}}
\newcommand{\DecValTok}[1]{\textcolor[rgb]{0.00,0.00,0.81}{#1}}
\newcommand{\DocumentationTok}[1]{\textcolor[rgb]{0.56,0.35,0.01}{\textbf{\textit{#1}}}}
\newcommand{\ErrorTok}[1]{\textcolor[rgb]{0.64,0.00,0.00}{\textbf{#1}}}
\newcommand{\ExtensionTok}[1]{#1}
\newcommand{\FloatTok}[1]{\textcolor[rgb]{0.00,0.00,0.81}{#1}}
\newcommand{\FunctionTok}[1]{\textcolor[rgb]{0.13,0.29,0.53}{\textbf{#1}}}
\newcommand{\ImportTok}[1]{#1}
\newcommand{\InformationTok}[1]{\textcolor[rgb]{0.56,0.35,0.01}{\textbf{\textit{#1}}}}
\newcommand{\KeywordTok}[1]{\textcolor[rgb]{0.13,0.29,0.53}{\textbf{#1}}}
\newcommand{\NormalTok}[1]{#1}
\newcommand{\OperatorTok}[1]{\textcolor[rgb]{0.81,0.36,0.00}{\textbf{#1}}}
\newcommand{\OtherTok}[1]{\textcolor[rgb]{0.56,0.35,0.01}{#1}}
\newcommand{\PreprocessorTok}[1]{\textcolor[rgb]{0.56,0.35,0.01}{\textit{#1}}}
\newcommand{\RegionMarkerTok}[1]{#1}
\newcommand{\SpecialCharTok}[1]{\textcolor[rgb]{0.81,0.36,0.00}{\textbf{#1}}}
\newcommand{\SpecialStringTok}[1]{\textcolor[rgb]{0.31,0.60,0.02}{#1}}
\newcommand{\StringTok}[1]{\textcolor[rgb]{0.31,0.60,0.02}{#1}}
\newcommand{\VariableTok}[1]{\textcolor[rgb]{0.00,0.00,0.00}{#1}}
\newcommand{\VerbatimStringTok}[1]{\textcolor[rgb]{0.31,0.60,0.02}{#1}}
\newcommand{\WarningTok}[1]{\textcolor[rgb]{0.56,0.35,0.01}{\textbf{\textit{#1}}}}
\usepackage{longtable,booktabs,array}
\usepackage{calc} % for calculating minipage widths
% Correct order of tables after \paragraph or \subparagraph
\usepackage{etoolbox}
\makeatletter
\patchcmd\longtable{\par}{\if@noskipsec\mbox{}\fi\par}{}{}
\makeatother
% Allow footnotes in longtable head/foot
\IfFileExists{footnotehyper.sty}{\usepackage{footnotehyper}}{\usepackage{footnote}}
\makesavenoteenv{longtable}
\usepackage{graphicx}
\makeatletter
\def\maxwidth{\ifdim\Gin@nat@width>\linewidth\linewidth\else\Gin@nat@width\fi}
\def\maxheight{\ifdim\Gin@nat@height>\textheight\textheight\else\Gin@nat@height\fi}
\makeatother
% Scale images if necessary, so that they will not overflow the page
% margins by default, and it is still possible to overwrite the defaults
% using explicit options in \includegraphics[width, height, ...]{}
\setkeys{Gin}{width=\maxwidth,height=\maxheight,keepaspectratio}
% Set default figure placement to htbp
\makeatletter
\def\fps@figure{htbp}
\makeatother
\setlength{\emergencystretch}{3em} % prevent overfull lines
\providecommand{\tightlist}{%
  \setlength{\itemsep}{0pt}\setlength{\parskip}{0pt}}
\setcounter{secnumdepth}{-\maxdimen} % remove section numbering
\ifLuaTeX
  \usepackage{selnolig}  % disable illegal ligatures
\fi
\IfFileExists{bookmark.sty}{\usepackage{bookmark}}{\usepackage{hyperref}}
\IfFileExists{xurl.sty}{\usepackage{xurl}}{} % add URL line breaks if available
\urlstyle{same}
\hypersetup{
  pdfauthor={Tomàs Ferrandis Moscardó},
  hidelinks,
  pdfcreator={LaTeX via pandoc}}

\title{ANÁLISIS UNIVARIABLE SOBRE PROBABILIDAD DE VOTO SEGÚN BARÓMETRO
DEL CIS DE JULIO 2023\\
(ESTUDIO 3415)}
\usepackage{etoolbox}
\makeatletter
\providecommand{\subtitle}[1]{% add subtitle to \maketitle
  \apptocmd{\@title}{\par {\large #1 \par}}{}{}
}
\makeatother
\subtitle{Técnicas de Investigación en Ciencias Políticas-UBU.\\
Práctica 1. Apartado 4.}
\author{Tomàs Ferrandis Moscardó}
\date{2023-03-22}

\begin{document}
\maketitle

{
\setcounter{tocdepth}{2}
\tableofcontents
}
\newpage

\hypertarget{introducciuxf3n}{%
\section{1 INTRODUCCIÓN}\label{introducciuxf3n}}

Este segundo estudio univariao se centrará en la variable PROBVOTO
(pregunta P1) del mismo cuestionario: \emph{Como Ud. sabe, el domingo 23
de julio se van a celebrar elecciones generales en España. Para comenzar
me gustaría que me dijera cuál es la probabilidad de que Ud. vaya a
votar. Para contestar utilice una escala de 0 a 10 en la que el 0
significa ``con toda seguridad no iría a votar'' y 10 ``con toda
seguridad, iría a votar''} por lo tanto se trata también de un
\emph{análisi univariado}.

Podemos ver que, pese a ser un variable cualitativa ordinal, esta será
manipulada como si fuese una cuantitativa para obtener más información
estadística. Se trata de una práctica habitual en ciencias sociales que
debe, en todo caso, quedar justificada.

\begin{quote}
NOTA: En este documento se ha activado la opción \emph{echo} de los
chuncks de R a fin de poder ser valorados académicamente. La creación
del fichero HTML, PDF o DOCX renderizado se haría tras la desactivación.
Solo llevan título de tablas de resultados que \textgreater{}
corresponderían a esta versión final y están maquetadas en Markdown.
\end{quote}

\begin{Shaded}
\begin{Highlighting}[]
\CommentTok{\# Para las librerías necesarias, se deben instalar los paquetes si no están ya instalados. }
\CommentTok{\# Ocultamos los mensajes.}
\FunctionTok{suppressMessages}\NormalTok{(\{}\ControlFlowTok{if}\NormalTok{ (}\SpecialCharTok{!}\FunctionTok{require}\NormalTok{(epiDisplay))\{}\FunctionTok{install.packages}\NormalTok{(}\StringTok{"epiDisplay"}\NormalTok{)\}}
\ControlFlowTok{if}\NormalTok{ (}\SpecialCharTok{!}\FunctionTok{require}\NormalTok{(e1071))\{}\FunctionTok{install.packages}\NormalTok{(}\StringTok{"e1071"}\NormalTok{)\}}
\ControlFlowTok{if}\NormalTok{ (}\SpecialCharTok{!}\FunctionTok{require}\NormalTok{(stats))\{}\FunctionTok{install.packages}\NormalTok{(}\StringTok{"stats"}\NormalTok{)\}}
\ControlFlowTok{if}\NormalTok{ (}\SpecialCharTok{!}\FunctionTok{require}\NormalTok{(tibble))\{}\FunctionTok{install.packages}\NormalTok{(}\StringTok{"tibble"}\NormalTok{)\}}
\NormalTok{\})}
\end{Highlighting}
\end{Shaded}

\hypertarget{obtenciuxf3n-de-datos}{%
\section{2 OBTENCIÓN DE DATOS}\label{obtenciuxf3n-de-datos}}

\hypertarget{fichero-y-datos-en-abierto}{%
\subsection{2.1 Fichero y Datos en
Abierto}\label{fichero-y-datos-en-abierto}}

\begin{Shaded}
\begin{Highlighting}[]
\CommentTok{\# El fichero .sav lo tenemos en una subcarpeta DATOS}
\FunctionTok{library}\NormalTok{(knitr)}
\NormalTok{nombreFichero}\OtherTok{\textless{}{-}}\StringTok{"datosjulio2023.sav"}
\NormalTok{directorioTrabajo}\OtherTok{\textless{}{-}}\FunctionTok{getwd}\NormalTok{()}
\NormalTok{rutaFichero}\OtherTok{\textless{}{-}}\FunctionTok{paste}\NormalTok{(directorioTrabajo,}\StringTok{"DATOS"}\NormalTok{,nombreFichero,}\AttributeTok{sep=}\StringTok{"/"}\NormalTok{)}
\end{Highlighting}
\end{Shaded}

\hypertarget{creaciuxf3n-de-la-base-de-datos-data-frame}{%
\subsection{\texorpdfstring{2.2 Creación de la base de datos (\emph{data
frame})}{2.2 Creación de la base de datos (data frame)}}\label{creaciuxf3n-de-la-base-de-datos-data-frame}}

A partir del fichero de tipo \emph{sav} se creará el \emph{data frame}
de R. En será la base de datos incial donde tendremos todos los datos
del barómetro en concreto del Estudio 3415.

\begin{Shaded}
\begin{Highlighting}[]
\NormalTok{df\_cisjulio}\OtherTok{\textless{}{-}}\NormalTok{haven}\SpecialCharTok{::}\FunctionTok{read\_sav}\NormalTok{(rutaFichero)}
\end{Highlighting}
\end{Shaded}

\begin{verbatim}
## Invalid date string (length=9): 25 038 23
\end{verbatim}

\hypertarget{consultar-el-cuestionario-y-el-diccionario-de-datos}{%
\subsection{2.3 Consultar el cuestionario y el diccionario de
datos}\label{consultar-el-cuestionario-y-el-diccionario-de-datos}}

Al margen de consultar al cuestionario del CIS, mediante R se puede
observar la variable para intentar deducir de qué tipo se trata.

Algunas consultas posibles podrían ser:

\begin{itemize}
\tightlist
\item
  Consultar algunos casos y los valores con etiquetas (en este caso) con
  \emph{head()} o \emph{tail()}
\item
  Usar \emph{table} que devuelve el número de casos de cada valor.
\item
  Agrupar todos los valores con \emph{unique}
\end{itemize}

\begin{Shaded}
\begin{Highlighting}[]
\FunctionTok{head}\NormalTok{(df\_cisjulio}\SpecialCharTok{$}\NormalTok{PROBVOTO,}\DecValTok{5}\NormalTok{) }\CommentTok{\# vemos solo 5 casos y los valores y etiquetas}
\end{Highlighting}
\end{Shaded}

\begin{verbatim}
## <labelled<double>[5]>: Escala de probabilidad de ir a votar (0-10) en las elecciones generales de 2023
## [1] 10 10 10  6 10
## 
## Labels:
##  value                                label
##      0 0 Con toda seguridad no iría a votar
##      1                                    1
##      2                                    2
##      3                                    3
##      4                                    4
##      5                                    5
##      6                                    6
##      7                                    7
##      8                                    8
##      9                                    9
##     10  10 Con toda seguridad, iría a votar
##     98                                 N.S.
##     99                                 N.C.
\end{verbatim}

\begin{Shaded}
\begin{Highlighting}[]
                             \CommentTok{\# de la variable puesto que es factor}

\FunctionTok{table}\NormalTok{(df\_cisjulio}\SpecialCharTok{$}\NormalTok{PROBVOTO) }
\end{Highlighting}
\end{Shaded}

\begin{verbatim}
## 
##    0    1    2    3    4    5    6    7    8    9   10   98   99 
##  249   35   31   32   23  183   75  117  303  574 7145   16   15
\end{verbatim}

\begin{Shaded}
\begin{Highlighting}[]
\FunctionTok{unique}\NormalTok{(df\_cisjulio}\SpecialCharTok{$}\NormalTok{PROBVOTO)}
\end{Highlighting}
\end{Shaded}

\begin{verbatim}
## <labelled<double>[13]>: Escala de probabilidad de ir a votar (0-10) en las elecciones generales de 2023
##  [1] 10  6  0 99  5  8  2  9  7  4  3 98  1
## 
## Labels:
##  value                                label
##      0 0 Con toda seguridad no iría a votar
##      1                                    1
##      2                                    2
##      3                                    3
##      4                                    4
##      5                                    5
##      6                                    6
##      7                                    7
##      8                                    8
##      9                                    9
##     10  10 Con toda seguridad, iría a votar
##     98                                 N.S.
##     99                                 N.C.
\end{verbatim}

\hypertarget{preparaciuxf3n-de-la-matriz-de-datos}{%
\section{3 PREPARACIÓN DE LA MATRIZ DE
DATOS}\label{preparaciuxf3n-de-la-matriz-de-datos}}

Una vez obtenida la matriz inicial de datos, se deberá limpiar y
simplificar esta base de datos inicial.

\hypertarget{limpieza-de-la-base-de-datos.}{%
\subsection{3.1 Limpieza de la base de
datos.}\label{limpieza-de-la-base-de-datos.}}

El data frame inicial contiene \texttt{ncol(df\_cisjulio)} columnas
correspondientes a todas las preguntas de la encuesta. Se puede
prescindir del resto de columnas del data frame innecesarias para
nuestro análisis univariado.

\begin{Shaded}
\begin{Highlighting}[]
\NormalTok{df\_cisProbVoto}\OtherTok{\textless{}{-}}\NormalTok{df\_cisjulio [, }\StringTok{"PROBVOTO"}\NormalTok{]}
\CommentTok{\#ncol(df\_cisjulio) \# devuelve el número de columnas del data frame. Lo usamos en nuestro}
\CommentTok{\#Markdown}
\end{Highlighting}
\end{Shaded}

Hemos pasado de \texttt{ncol(df\_cisjulio)} a 1 sola columna.

\hypertarget{recodificciuxf3n}{%
\subsection{3.2 Recodificción}\label{recodificciuxf3n}}

En estecaso hay pocos valores válidos (11) como es habitual en una
variable cualitativa, no se va a grupar valores pero si se debe decidir
qué valores no sol válidos en la estadística y excluirlos.

\hypertarget{duplicamos-la-columna}{%
\subsubsection{Duplicamos la columna}\label{duplicamos-la-columna}}

Antes de modificar cualquier campo, se debe hacer un dupicado de este
(columna del data frame) y trabajar sobre él.

\begin{Shaded}
\begin{Highlighting}[]
\NormalTok{df\_cisProbVoto}\SpecialCharTok{$}\NormalTok{recPROBVOTO}\OtherTok{\textless{}{-}}\NormalTok{df\_cisProbVoto}\SpecialCharTok{$}\NormalTok{PROBVOTO}
\end{Highlighting}
\end{Shaded}

\hypertarget{valores-vuxe1lidos-y-valores-no-vuxe1lidos}{%
\subsubsection{Valores válidos y valores no
válidos}\label{valores-vuxe1lidos-y-valores-no-vuxe1lidos}}

Valores no válidos: * 98 N.S. * 99 N.C. Valores válidos: 0..10

Los valores no válidos tienen un doble tratamiento:

\begin{itemize}
\tightlist
\item
  Valorar su importancia
\item
  Excluirlos de los cálculos
\end{itemize}

\emph{Importancia de los valores perdidos}

Para valorar la incidencia o relevancia de los valores no válidos se
puede de qué porcentaje de casos de la muestra se trata.

Vemos que tenemos 31 de un total de 8798. Un porcentaje inferior a un
0.4 \% de los casos y, por tanto, insignificante.

\begin{quote}
Obtenemos estos valores con

\texttt{\#r\ nrow(df\_cisProbVoto{[}df\_cisProbVoto\$recPROBVOTO\textgreater{}=98,{]})}

\texttt{\#r\ nrow(df\_cisProbVoto)}

\texttt{\#r\ nrow(df\_cisProbVoto{[}df\_cisProbVoto\$recPROBVOTO\textgreater{}=98,{]})/nrow(df\_cisProbVoto)*100}
\end{quote}

\emph{Asignar NA para excluir en el análisis}

R permite excluir en sus funciones estadísticas los valores perdidos con
parametrizaciones al estilo \emph{MISSING=TRUE} o \emph{na.rm = TRUE}.
Para ello, previamente se tiene que asignar el valor NA a estos campos.

\begin{Shaded}
\begin{Highlighting}[]
\NormalTok{df\_cisProbVoto}\SpecialCharTok{$}\NormalTok{recPROBVOTO[df\_cisProbVoto}\SpecialCharTok{$}\NormalTok{recPROBVOTO}\SpecialCharTok{\textgreater{}=}\DecValTok{98}\NormalTok{]}\OtherTok{\textless{}{-}}\ConstantTok{NA}
\end{Highlighting}
\end{Shaded}

\hypertarget{distribuciuxf3n-de-frecuencias}{%
\section{4. DISTRIBUCIÓN DE
FRECUENCIAS}\label{distribuciuxf3n-de-frecuencias}}

En esta apartado se verá el uso de tablas de distribución de frecuencias
y gráficos, métodos fundamnetales que nos muestran as constituyen una
primera forma de dar cuenta de la información sobre las variables.

\hypertarget{gruxe1fico-de-barras}{%
\subsection{4.1 Gráfico de barras}\label{gruxe1fico-de-barras}}

\begin{Shaded}
\begin{Highlighting}[]
\NormalTok{tablaGrafico }\OtherTok{\textless{}{-}} \FunctionTok{tab1}\NormalTok{(df\_cisProbVoto}\SpecialCharTok{$}\NormalTok{recPROBVOTO, }\AttributeTok{cum.percent =} \ConstantTok{TRUE}\NormalTok{, }
             \AttributeTok{bar.values =} \StringTok{"percent"}\NormalTok{, }\AttributeTok{missing =} \ConstantTok{FALSE}\NormalTok{,}
             \AttributeTok{xlab=}\StringTok{"\% probabilidad"}\NormalTok{,}\AttributeTok{decimal=}\DecValTok{1}\NormalTok{,}
             \AttributeTok{main=}\StringTok{"Gráfico 1. Probabilidad de votar"}\NormalTok{)}
\end{Highlighting}
\end{Shaded}

\includegraphics{probabilidadVoto_files/figure-latex/graficoBarras-1.pdf}

\hypertarget{tabla-de-frecuencias}{%
\subsection{4.2 Tabla de frecuencias}\label{tabla-de-frecuencias}}

En la tabla de frecuencias siguiente se muestra el total casos por cada
valor, la frecuencia en que aparecen y la frecuencia acumulada. Las
frecuencias puede calcularse respecto al totas de casos o respecto al
total de casos válidos. En este caso, como ya se ha apuntado
anteriormente, la diferencia es insignificante.

\begin{Shaded}
\begin{Highlighting}[]
\CommentTok{\# Mediante el atributo output.table se exporta una tabla}
\NormalTok{tabla}\OtherTok{\textless{}{-}}\NormalTok{tablaGrafico}\SpecialCharTok{$}\NormalTok{output.table}
\CommentTok{\#Para mejorar la comprensión se puede cambiar los nombres de las columnas e imprimir con MD}
\FunctionTok{colnames}\NormalTok{(tabla)}\OtherTok{\textless{}{-}}\FunctionTok{c}\NormalTok{(}\StringTok{"Frecuencia"}\NormalTok{,}\StringTok{"\%"}\NormalTok{,}\StringTok{"\%\_acumulado"}\NormalTok{, }\StringTok{"\%\_válido"}\NormalTok{,}\StringTok{"\%\_acumulado válido"}\NormalTok{)}
\NormalTok{knitr}\SpecialCharTok{::}\FunctionTok{kable}\NormalTok{(tabla,}\AttributeTok{column.width=}\FunctionTok{c}\NormalTok{(}\StringTok{"10\%"}\NormalTok{,}\StringTok{"20\%"}\NormalTok{,}\StringTok{"20\%"}\NormalTok{,}\StringTok{"20\%"}\NormalTok{,}\StringTok{"20\%"}\NormalTok{),}\AttributeTok{caption=}\StringTok{"Tabla1.}
\StringTok{              Tabla de distribución de frecuencias"}\NormalTok{) }
\end{Highlighting}
\end{Shaded}

\begin{longtable}[]{@{}lrrrrr@{}}
\caption{Tabla1. Tabla de distribución de frecuencias}\tabularnewline
\toprule\noalign{}
& Frecuencia & \% & \%\_acumulado & \%\_válido & \%\_acumulado válido \\
\midrule\noalign{}
\endfirsthead
\toprule\noalign{}
& Frecuencia & \% & \%\_acumulado & \%\_válido & \%\_acumulado válido \\
\midrule\noalign{}
\endhead
\bottomrule\noalign{}
\endlastfoot
0 & 249 & 2.8 & 2.8 & 2.8 & 2.8 \\
1 & 35 & 0.4 & 3.2 & 0.4 & 3.2 \\
2 & 31 & 0.4 & 3.6 & 0.4 & 3.6 \\
3 & 32 & 0.4 & 3.9 & 0.4 & 4.0 \\
4 & 23 & 0.3 & 4.2 & 0.3 & 4.2 \\
5 & 183 & 2.1 & 6.3 & 2.1 & 6.3 \\
6 & 75 & 0.9 & 7.1 & 0.9 & 7.2 \\
7 & 117 & 1.3 & 8.5 & 1.3 & 8.5 \\
8 & 303 & 3.4 & 11.9 & 3.5 & 12.0 \\
9 & 574 & 6.5 & 18.4 & 6.5 & 18.5 \\
10 & 7145 & 81.2 & 99.6 & 81.5 & 100.0 \\
NA & 31 & 0.4 & 100.0 & 0.0 & 100.0 \\
Total & 8798 & 100.0 & 100.0 & 100.0 & 100.0 \\
\end{longtable}

\begin{Shaded}
\begin{Highlighting}[]
\CommentTok{\# print(tabla)  \# En r estrictamente}

\CommentTok{\# Para calcular las frecuencias directamente:}
\CommentTok{\# round(prop.table(table(df\_cisProbVoto$recPROBVOTO, useNA="no"))*100,1)}
\end{Highlighting}
\end{Shaded}

\hypertarget{tratamiento-como-variable-cuantitativa}{%
\section{5 TRATAMIENTO COMO VARIABLE
CUANTITATIVA}\label{tratamiento-como-variable-cuantitativa}}

Aunque la probabilidad de votar en este cuestionario se trata de una
variable \emph{cualitativa ordinal}, se podría asumir que los valores
válidos (0..10) representan una magnitud que admite operaciones
artiméticas y darle un tratamiento de \emph{variable cuantitativa}. En
este contexto podría entenderse como de variable \emph{razón} si se
asume el valor 0 como la ausencia absoluta de intención de ir a votar.
No obstante se trata de una interpretación que no debe sacarse del
contexto del análisis concreto.

Este planteamiento, excepcional, se justifica por la necesidad de
obtener los valores de las \emph{medidas de tendencia central} más allá
de la \emph{moda}

\hypertarget{medidas-de-tendencia-central}{%
\subsection{5.1 Medidas de tendencia
central}\label{medidas-de-tendencia-central}}

\hypertarget{funciuxf3n-summary}{%
\subsubsection{\texorpdfstring{Función
\emph{summary}}{Función summary}}\label{funciuxf3n-summary}}

Esta función nos aporta las medidas de posición central (moda, media y
mediana) y las de posición no central (valores extremos y cuartiles). El
segundo cuartil equivale a la mediana.

\begin{Shaded}
\begin{Highlighting}[]
\FunctionTok{summary}\NormalTok{(df\_cisProbVoto}\SpecialCharTok{$}\NormalTok{recPROBVOTO)}
\end{Highlighting}
\end{Shaded}

\begin{verbatim}
##    Min. 1st Qu.  Median    Mean 3rd Qu.    Max.    NA's 
##   0.000  10.000  10.000   9.297  10.000  10.000      31
\end{verbatim}

\hypertarget{media-mediana-y-moda}{%
\subsubsection{Media, Mediana y Moda}\label{media-mediana-y-moda}}

Estas medidas de posición central se pueden obtener mediante funciones
específicas con posibilidad de especificar opciones en su cálculo
(borrar NA, número de decimales\ldots) y guardar su valor en una
variable para su posterior uso.

\textbf{Media} Promedio aritmético de todos los valores de la muestra.

\begin{Shaded}
\begin{Highlighting}[]
\NormalTok{media}\OtherTok{\textless{}{-}}\FunctionTok{round}\NormalTok{(}\FunctionTok{mean}\NormalTok{(df\_cisProbVoto}\SpecialCharTok{$}\NormalTok{recPROBVOTO, }\AttributeTok{na.rm =} \ConstantTok{TRUE}\NormalTok{),}\DecValTok{1}\NormalTok{)}
\end{Highlighting}
\end{Shaded}

\textbf{Mediana} Valor de la muestra que divide la muestra en dos
mitades o grupos con igual cantidad de casos, siendo uno de los grupos
el formado por casoa con un valor superior y el otro grupo el formado
por los casos con valor inferior.

\begin{Shaded}
\begin{Highlighting}[]
\NormalTok{mediana}\OtherTok{\textless{}{-}}\FunctionTok{median}\NormalTok{(df\_cisProbVoto}\SpecialCharTok{$}\NormalTok{recPROBVOTO, }\AttributeTok{na.rm =} \ConstantTok{TRUE}\NormalTok{,}\DecValTok{1}\NormalTok{)}
\end{Highlighting}
\end{Shaded}

\begin{quote}
La opción \emph{na.rm = TRUE} indica no se debe considerar los valores
no validos: remove NA Redondeamos con la función \emph{round}, como es
habitual en ciencias sociales a 1 decimal
\end{quote}

\textbf{Moda} La moda es el valor de la muestra que más se repite.
\textgreater{} En R no existe una función que devuelva la moda
directamente similar a las vistas para mediana \emph{median()} y para la
media \emph{mean()}. Se debe usar una diseñada expresamente.

\emph{Función moda}

\begin{Shaded}
\begin{Highlighting}[]
\CommentTok{\#declaración y desarollo de la función de usuario}
\NormalTok{moda}\OtherTok{\textless{}{-}} \ControlFlowTok{function}\NormalTok{(v) \{}
\NormalTok{uniqv }\OtherTok{\textless{}{-}} \FunctionTok{unique}\NormalTok{(v)}
\NormalTok{uniqv[}\FunctionTok{which.max}\NormalTok{(}\FunctionTok{tabulate}\NormalTok{(}\FunctionTok{match}\NormalTok{(v, uniqv)))]}
\NormalTok{\}}
\end{Highlighting}
\end{Shaded}

Una vez definida la función se puede llamar como las incorporadas en las
librerías.

\begin{Shaded}
\begin{Highlighting}[]
\CommentTok{\#Se llama a la función previamente desarrollada}
\NormalTok{moda}\OtherTok{\textless{}{-}}\FunctionTok{moda}\NormalTok{(df\_cisProbVoto}\SpecialCharTok{$}\NormalTok{recPROBVOTO) }
\end{Highlighting}
\end{Shaded}

\textbf{Resumen de valores de posición central}

\begin{itemize}
\tightlist
\item
  La media es 9.3
\item
  La mediana es 10
\item
  La moda es 10
\end{itemize}

\textbf{Conclusiones}

La \emph{media aritmética}: 9.3 está muy influida la altísima frecuencia
de los valores más altos, sobretodo del máximo. Pese que la
\emph{mediana} nos de como resultado 10, el total de casos con valor
inferior no llega a ser el 50\%.

La moda sin duda es el valor 10 a la vista del porcentaje reflejado en
la tabla de frecuencia.

\hypertarget{comparaciuxf3n-entre-moda-media-y-mediana.}{%
\subsection{5.2 Comparación entre Moda, Media y
Mediana.}\label{comparaciuxf3n-entre-moda-media-y-mediana.}}

\textbf{Histograma}

El histograma mostrará una asimetría positiva (sesgo a la izquierda)
donde la Media y a Moda coinciden y la Media es el valor más bajo
\textbf{X\textless Me=Mo}

\begin{Shaded}
\begin{Highlighting}[]
\CommentTok{\# Eliminar filas con NA}
\NormalTok{df\_cisProbVotoSinNA }\OtherTok{\textless{}{-}}\NormalTok{ df\_cisProbVoto[}\FunctionTok{complete.cases}\NormalTok{(df\_cisProbVoto}\SpecialCharTok{$}\NormalTok{recPROBVOTO), ]}

\FunctionTok{hist}\NormalTok{(df\_cisProbVoto}\SpecialCharTok{$}\NormalTok{recPROBVOTO, }\AttributeTok{col =} \StringTok{"lightblue"}\NormalTok{, }\AttributeTok{xlab =} \StringTok{"Probabilidad de voto"}\NormalTok{, }
     \AttributeTok{ylab =} \StringTok{"Frecuencia"}\NormalTok{,}
     \AttributeTok{main =} \StringTok{"Gráfico 2. Histograma de la probabilidad de votar"}\NormalTok{)}
\FunctionTok{lines}\NormalTok{(}\FunctionTok{density}\NormalTok{(df\_cisProbVotoSinNA}\SpecialCharTok{$}\NormalTok{recPROBVOTO), }\AttributeTok{col =} \StringTok{"red"}\NormalTok{, }\AttributeTok{lwd =} \DecValTok{1}\NormalTok{)}
\end{Highlighting}
\end{Shaded}

\includegraphics{probabilidadVoto_files/figure-latex/historiograma-1.pdf}
\#\#\# Conclusión sobre las medidas de posición central

Debido al sesgo ( a la derecha ) la mediana tener un valor mayor que la
Moda y menor que la media. La Mediada debería ser una medida de posición
central mejor que la media que quedaría que la Media distorsionada por
los extremos o la Moda definida por la repetición de un valor. En
nuestro caso, en cambio, la Mediana coincide con la Moda.

\hypertarget{medidas-de-posiciuxf3n-no-central}{%
\subsection{5.3 Medidas de posición no
central}\label{medidas-de-posiciuxf3n-no-central}}

\hypertarget{valores-extremos-muxe1ximo-y-muxednimo}{%
\subsubsection{Valores extremos: Máximo y
Mínimo}\label{valores-extremos-muxe1ximo-y-muxednimo}}

\begin{Shaded}
\begin{Highlighting}[]
\NormalTok{maximo}\OtherTok{\textless{}{-}}\FunctionTok{max}\NormalTok{(df\_cisProbVoto}\SpecialCharTok{$}\NormalTok{recPROBVOTO, }\AttributeTok{na.rm =} \ConstantTok{TRUE}\NormalTok{)}

\NormalTok{minimo}\OtherTok{\textless{}{-}}\FunctionTok{min}\NormalTok{(df\_cisProbVoto}\SpecialCharTok{$}\NormalTok{recPROBVOTO, }\AttributeTok{na.rm =} \ConstantTok{TRUE}\NormalTok{)}
\end{Highlighting}
\end{Shaded}

Los valores extremos son:

\begin{itemize}
\tightlist
\item
  El valor MÁXIMO es: 10
\item
  El valor MÍNIMO es: 0
\end{itemize}

Al tratarse de una pregunta cerrada con una ordenación entre las
respuestas (variable cualitativa ordinal), estos valores indican el par
de respuestas más dispares. La coincidencia de máximo y mínimo con las
opciones más extremas implica que ha habido casos donde el encuestado ha
elegido estos valores, simplemente.

\hypertarget{cuartiles-y-percentiles}{%
\subsubsection{Cuartiles y percentiles}\label{cuartiles-y-percentiles}}

En una distribución de valores de una muestra tan sesgada, un cuartil no
da demasiada información. La inmensa mayoría de valores son 10 y ek
resto son tratados como valores atípicos ( 1,5 veces la caja).

\begin{Shaded}
\begin{Highlighting}[]
\CommentTok{\# Crear el diagrama de caja}
\FunctionTok{boxplot}\NormalTok{(df\_cisProbVotoSinNA}\SpecialCharTok{$}\NormalTok{recPROBVOTO, }
        \AttributeTok{main =} \StringTok{"Grafico 3. Cuartiles probabilidad de votar"}\NormalTok{,}
        \AttributeTok{ylab =} \StringTok{"Valores de 0 a 10"}\NormalTok{, }
        \AttributeTok{col =} \StringTok{"lightblue"}\NormalTok{)}

\CommentTok{\# Calcular los cuartiles}
\NormalTok{cuartiles }\OtherTok{\textless{}{-}} \FunctionTok{quantile}\NormalTok{(df\_cisProbVotoSinNA}\SpecialCharTok{$}\NormalTok{recPROBVOTO, }\AttributeTok{probs =} \FunctionTok{c}\NormalTok{(}\FloatTok{0.25}\NormalTok{, }\FloatTok{0.5}\NormalTok{, }\FloatTok{0.75}\NormalTok{))}

\CommentTok{\# Agregar las líneas de los los cuartiles}
\FunctionTok{abline}\NormalTok{(}\AttributeTok{h =}\NormalTok{ cuartiles, }\AttributeTok{col =} \FunctionTok{c}\NormalTok{(}\StringTok{"red"}\NormalTok{, }\StringTok{"green"}\NormalTok{, }\StringTok{"blue"}\NormalTok{))}
\end{Highlighting}
\end{Shaded}

\includegraphics{probabilidadVoto_files/figure-latex/cuartiles-1.pdf}

Se puede usar los percentiles para analizar mejor la información.

\begin{Shaded}
\begin{Highlighting}[]
\NormalTok{percentiles}\OtherTok{\textless{}{-}}\FunctionTok{quantile}\NormalTok{(df\_cisProbVotoSinNA, }\FunctionTok{c}\NormalTok{(.}\DecValTok{10}\NormalTok{,.}\DecValTok{20}\NormalTok{,.}\DecValTok{30}\NormalTok{,.}\DecValTok{40}\NormalTok{,.}\DecValTok{50}\NormalTok{,.}\DecValTok{60}\NormalTok{,.}\DecValTok{70}\NormalTok{,.}\DecValTok{80}\NormalTok{,.}\DecValTok{90}\NormalTok{), }\AttributeTok{na.rm =} \ConstantTok{TRUE}\NormalTok{)}
\FunctionTok{boxplot}\NormalTok{(df\_cisProbVotoSinNA}\SpecialCharTok{$}\NormalTok{recPROBVOTO, }\AttributeTok{main =} \StringTok{"Grafico 3. Percentiles probabilidad de votar"}\NormalTok{,}
        \AttributeTok{ylab =} \StringTok{"Valores de 0 a 10"}\NormalTok{, }\AttributeTok{col =} \StringTok{"lightblue"}\NormalTok{)}
\CommentTok{\# Agregar las líneas de los los cuartiles}
\FunctionTok{abline}\NormalTok{(}\AttributeTok{h =}\NormalTok{ percentiles, }\AttributeTok{col =} \StringTok{"red"}\NormalTok{)}
\end{Highlighting}
\end{Shaded}

\includegraphics{probabilidadVoto_files/figure-latex/percentiles0-1.pdf}

\begin{Shaded}
\begin{Highlighting}[]
\NormalTok{percentiles}\OtherTok{\textless{}{-}}\FunctionTok{quantile}\NormalTok{(df\_cisProbVotoSinNA, }\FunctionTok{c}\NormalTok{(.}\DecValTok{02}\NormalTok{,.}\DecValTok{03}\NormalTok{,.}\DecValTok{04}\NormalTok{,.}\DecValTok{05}\NormalTok{,.}\DecValTok{06}\NormalTok{,.}\DecValTok{07}\NormalTok{,.}\DecValTok{08}\NormalTok{,.}\DecValTok{09}\NormalTok{,.}\DecValTok{10}\NormalTok{,.}\DecValTok{20}\NormalTok{,.}\DecValTok{30}\NormalTok{,.}\DecValTok{40}\NormalTok{,}
\NormalTok{                                             .}\DecValTok{50}\NormalTok{,.}\DecValTok{60}\NormalTok{,.}\DecValTok{70}\NormalTok{,.}\DecValTok{80}\NormalTok{,.}\DecValTok{90}\NormalTok{), }\AttributeTok{na.rm =} \ConstantTok{TRUE}\NormalTok{)}
\FunctionTok{boxplot}\NormalTok{(df\_cisProbVotoSinNA}\SpecialCharTok{$}\NormalTok{recPROBVOTO, }\AttributeTok{main =} \StringTok{"Grafico 3. Percentiles probabilidad de votar"}\NormalTok{,}
        \AttributeTok{ylab =} \StringTok{"Valores de 0 a 10"}\NormalTok{, }\AttributeTok{col =} \StringTok{"lightblue"}\NormalTok{)}
\CommentTok{\# Agregar las líneas de los los cuartiles}
\FunctionTok{abline}\NormalTok{(}\AttributeTok{h =}\NormalTok{ percentiles, }\AttributeTok{col =} \FunctionTok{c}\NormalTok{(}\StringTok{"red"}\NormalTok{, }\StringTok{"green"}\NormalTok{, }\StringTok{"blue"}\NormalTok{,}\StringTok{"red"}\NormalTok{, }\StringTok{"green"}\NormalTok{,}\StringTok{"blue"}\NormalTok{,}\StringTok{"red"}\NormalTok{, }\StringTok{"green"}\NormalTok{,}
                \StringTok{"blue"}\NormalTok{,}\StringTok{"red"}\NormalTok{, }\StringTok{"green"}\NormalTok{, }\StringTok{"blue"}\NormalTok{,}\StringTok{"red"}\NormalTok{, }\StringTok{"green"}\NormalTok{, }\StringTok{"blue"}\NormalTok{,}\StringTok{"red"}\NormalTok{, }\StringTok{"green"}\NormalTok{, }\StringTok{"blue"}\NormalTok{))}
\end{Highlighting}
\end{Shaded}

\includegraphics{probabilidadVoto_files/figure-latex/percentiles-1.pdf}
\emph{Conclusión:} SOlo un porcentaje inferior al 9\% de la muestra ha
respuesto un valor inferior al `8'. Algo que se puede ver también en la
columna de frecuencias acumuladas ( Tabla de frecuencia) donde el valor
8 acumula el 8,5\%.

\hypertarget{medidas-de-dispersiuxf3n}{%
\subsection{5.4 Medidas de dispersión}\label{medidas-de-dispersiuxf3n}}

Se procedera a obtener las medidas de varibilidad o dispersión que
informan del comportamiento de la variable y complementan la información
sobre la centralidad.

\hypertarget{varianza-o-desviaciuxf3n-tuxedpica}{%
\subsubsection{Varianza o desviación
típica}\label{varianza-o-desviaciuxf3n-tuxedpica}}

\begin{Shaded}
\begin{Highlighting}[]
\NormalTok{varianza}\OtherTok{\textless{}{-}}\FunctionTok{var}\NormalTok{(df\_cisProbVotoSinNA}\SpecialCharTok{$}\NormalTok{recPROBVOTO) }
\NormalTok{varianza}\OtherTok{\textless{}{-}}\FunctionTok{var}\NormalTok{(df\_cisProbVoto}\SpecialCharTok{$}\NormalTok{recPROBVOTO,}\AttributeTok{na.rm =} \ConstantTok{TRUE}\NormalTok{)  }\CommentTok{\# Si en el data frame aún están }
\CommentTok{\# los valores perdidos.}
\end{Highlighting}
\end{Shaded}

La desviación típica es: 4.2

\hypertarget{desviaciuxf3n-estuxe1ndar}{%
\subsubsection{Desviación Estándar}\label{desviaciuxf3n-estuxe1ndar}}

\begin{Shaded}
\begin{Highlighting}[]
\NormalTok{desviacionEstandar}\OtherTok{\textless{}{-}}\FunctionTok{sd}\NormalTok{(df\_cisProbVoto}\SpecialCharTok{$}\NormalTok{recPROBVOTO, }\AttributeTok{na.rm =} \ConstantTok{TRUE}\NormalTok{)}
\end{Highlighting}
\end{Shaded}

La desviación estándar es: 2

\hypertarget{rango-intercuatuxedlico}{%
\subsubsection{Rango intercuatílico}\label{rango-intercuatuxedlico}}

\begin{Shaded}
\begin{Highlighting}[]
\NormalTok{iqr}\OtherTok{\textless{}{-}}\FunctionTok{IQR}\NormalTok{(df\_cisProbVoto}\SpecialCharTok{$}\NormalTok{recPROBVOTO, }\AttributeTok{na.rm =} \ConstantTok{TRUE}\NormalTok{)}
\end{Highlighting}
\end{Shaded}

El rango intercuartílico es 0

\hypertarget{curtosis}{%
\subsubsection{Curtosis}\label{curtosis}}

El coeficiente de curtosis indica el nivel de apuntamiento achatamiento
que presenta una distribución de valores. Solo tien sentidos en
distribuciones unimodales y simétricas. No es este nuestro caso.

\begin{Shaded}
\begin{Highlighting}[]
\NormalTok{coeficienteKurtosis}\OtherTok{\textless{}{-}}\FunctionTok{kurtosis}\NormalTok{(df\_cisProbVoto}\SpecialCharTok{$}\NormalTok{recPROBVOTO, }\AttributeTok{na.rm =} \ConstantTok{TRUE}\NormalTok{)}
\end{Highlighting}
\end{Shaded}

El coeficiente de curtosis es: 11.9497589

\hypertarget{coeficiente-de-asimetruxeda}{%
\subsubsection{Coeficiente de
Asimetría}\label{coeficiente-de-asimetruxeda}}

\begin{Shaded}
\begin{Highlighting}[]
\CommentTok{\#skewness(df\_cisProbVoto$recPROBVOTO)}

\NormalTok{coeficienteAsimetria}\OtherTok{\textless{}{-}}\FunctionTok{skewness}\NormalTok{(df\_cisProbVoto}\SpecialCharTok{$}\NormalTok{recPROBVOTO, }\AttributeTok{na.rm =} \ConstantTok{TRUE}\NormalTok{)}
\end{Highlighting}
\end{Shaded}

El coeficiente de asimetŕia es: -3.5101093

\hypertarget{grafico-de-densidad}{%
\subsubsection{Grafico de densidad}\label{grafico-de-densidad}}

\begin{Shaded}
\begin{Highlighting}[]
\FunctionTok{plot}\NormalTok{(df\_cisProbVoto}\SpecialCharTok{$}\NormalTok{recPROBVOTO,}\AttributeTok{main =} \StringTok{"Gráfico de densidad"}\NormalTok{,}\AttributeTok{xlab =} \StringTok{"Casos"}\NormalTok{,}
     \AttributeTok{ylab =} \StringTok{"Probabilidad de voto"}\NormalTok{)}
\end{Highlighting}
\end{Shaded}

\includegraphics{probabilidadVoto_files/figure-latex/graficoDensidad-1.pdf}

\newpage

\hypertarget{resumen-valores}{%
\section{6 RESUMEN VALORES}\label{resumen-valores}}

\begin{longtable}[]{@{}lr@{}}
\toprule\noalign{}
MEDICIÓN & \\
\midrule\noalign{}
\endhead
\bottomrule\noalign{}
\endlastfoot
\textbf{POSICION CENTRAL (Medidas de tendencia central)} & \\
MODA & 10 \\
MEDIANA & 10 \\
MEDIA & 9.3 \\
\textbf{POSICIÓN (Medidas de posición no central)} & \\
MÁXIMO & 10 \\
MÍNIMO & 0 \\
PERCENTIL 25 & 10 \\
PERCENTIL 50 & 10 \\
PERCENTIL 75 & 10 \\
\textbf{MEDIDAS DE DISPERSIÓN} & \\
RANGO & 10 \\
RANGO INTERCUARTIL & 0 \\
DESVIACIÓN TÍPICA & 4.2 \\
DESVIACIÓN TÍPICA & 2 \\
\textbf{MEDIDAS DE FORMA} & \\
COEFICIENTE DE ASIMETRÍA & -3.5 \\
COEFICIENTE DE KURTOSIS & 12 \\
\end{longtable}

\end{document}
