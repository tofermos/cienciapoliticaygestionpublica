% Options for packages loaded elsewhere
\PassOptionsToPackage{unicode}{hyperref}
\PassOptionsToPackage{hyphens}{url}
%
\documentclass[
]{article}
\usepackage{amsmath,amssymb}
\usepackage{iftex}
\ifPDFTeX
  \usepackage[T1]{fontenc}
  \usepackage[utf8]{inputenc}
  \usepackage{textcomp} % provide euro and other symbols
\else % if luatex or xetex
  \usepackage{unicode-math} % this also loads fontspec
  \defaultfontfeatures{Scale=MatchLowercase}
  \defaultfontfeatures[\rmfamily]{Ligatures=TeX,Scale=1}
\fi
\usepackage{lmodern}
\ifPDFTeX\else
  % xetex/luatex font selection
\fi
% Use upquote if available, for straight quotes in verbatim environments
\IfFileExists{upquote.sty}{\usepackage{upquote}}{}
\IfFileExists{microtype.sty}{% use microtype if available
  \usepackage[]{microtype}
  \UseMicrotypeSet[protrusion]{basicmath} % disable protrusion for tt fonts
}{}
\makeatletter
\@ifundefined{KOMAClassName}{% if non-KOMA class
  \IfFileExists{parskip.sty}{%
    \usepackage{parskip}
  }{% else
    \setlength{\parindent}{0pt}
    \setlength{\parskip}{6pt plus 2pt minus 1pt}}
}{% if KOMA class
  \KOMAoptions{parskip=half}}
\makeatother
\usepackage{xcolor}
\usepackage[margin=1in]{geometry}
\usepackage{longtable,booktabs,array}
\usepackage{calc} % for calculating minipage widths
% Correct order of tables after \paragraph or \subparagraph
\usepackage{etoolbox}
\makeatletter
\patchcmd\longtable{\par}{\if@noskipsec\mbox{}\fi\par}{}{}
\makeatother
% Allow footnotes in longtable head/foot
\IfFileExists{footnotehyper.sty}{\usepackage{footnotehyper}}{\usepackage{footnote}}
\makesavenoteenv{longtable}
\usepackage{graphicx}
\makeatletter
\def\maxwidth{\ifdim\Gin@nat@width>\linewidth\linewidth\else\Gin@nat@width\fi}
\def\maxheight{\ifdim\Gin@nat@height>\textheight\textheight\else\Gin@nat@height\fi}
\makeatother
% Scale images if necessary, so that they will not overflow the page
% margins by default, and it is still possible to overwrite the defaults
% using explicit options in \includegraphics[width, height, ...]{}
\setkeys{Gin}{width=\maxwidth,height=\maxheight,keepaspectratio}
% Set default figure placement to htbp
\makeatletter
\def\fps@figure{htbp}
\makeatother
\setlength{\emergencystretch}{3em} % prevent overfull lines
\providecommand{\tightlist}{%
  \setlength{\itemsep}{0pt}\setlength{\parskip}{0pt}}
\setcounter{secnumdepth}{-\maxdimen} % remove section numbering
\ifLuaTeX
  \usepackage{selnolig}  % disable illegal ligatures
\fi
\IfFileExists{bookmark.sty}{\usepackage{bookmark}}{\usepackage{hyperref}}
\IfFileExists{xurl.sty}{\usepackage{xurl}}{} % add URL line breaks if available
\urlstyle{same}
\hypersetup{
  pdftitle={EL SISTEMA DE PARTIDOS ALEMÁN TRAS LAS ELECCIONES DEL 2021},
  pdfauthor={Tomàs Ferrandis Moscardó},
  hidelinks,
  pdfcreator={LaTeX via pandoc}}

\title{EL SISTEMA DE PARTIDOS ALEMÁN TRAS LAS ELECCIONES DEL 2021}
\author{Tomàs Ferrandis Moscardó}
\date{2021-12-10}

\begin{document}
\maketitle

{
\setcounter{tocdepth}{2}
\tableofcontents
}
\hypertarget{resumen}{%
\section{1. RESUMEN}\label{resumen}}

El presente trabajo es una parte de un examen Take Home de la asignatura
de \emph{Introducción a la Ciencia política}. en él se analiza el
sistema de partidos políticos alemán tras las elecciones federales de
Alemania en 2021.

\hypertarget{introducciuxf3n}{%
\section{2. INTRODUCCIÓN}\label{introducciuxf3n}}

Las elecciones federales de Alemania de 2021 se celebraron el 26 de
septiembre de 2021 en las que se eligieron 736 diputados para el
Bundestag ( cámara baja alemana ).

El Parlamento alemán cuenta con un mínimo de 598 escaños. De éstos, 299
son adjudicados a candidatos de los länders ( estados ) elegidos
directamente en el primer voto de la papeleta. Los otros 299 escaños se
reparten entre los partidos elegidos a través del segundo voto basado en
listas cerradas. No obstante, se pueden añadir diputados posteriormente
para asegurar una proporcionalidad en los votos nacionales. En 2017
tendríamos un Bundestag de 709 escaños y en 2021, de 736.

\hypertarget{consideraciones-previas}{%
\subsection{Consideraciones previas}\label{consideraciones-previas}}

En el presente trabajo, a efectos de cálculos para la obtención de
indicadores parlamentarios, nos basaremos en el total de diputados
obtenido de las dos papeletas. Para cálculos electorales ( porcentaje de
votos ) consideraremos los votos a las listas cerradas ( segunda
papeleta ).

En ambos casos tendremos en cuenta los partidos que han obtenido
representación en las elecciones del 2021 o las anteriores del 2017. En
cuanto a los partidos de la democracia-cristiana se consideraran sumados
los resultados del partido de ámbito nacional ( CDU ) y el de Baviera (
CSU ).

\hypertarget{fragmentaciuxf3n-parlamentaria}{%
\section{3. FRAGMENTACIÓN
PARLAMENTARIA}\label{fragmentaciuxf3n-parlamentaria}}

La fragmentación es un valor entre 0 y 1 que indica el número de
partidos y el tamaño relativo de éstos. Es decir, nos muestra cómo de
fraccionado está el sistema de partidos. Cuando los valores se acercan a
1 se tiende a un sistema más fraccionado mientras que si se acercan a 0,
tenderíamos a un partido pocas opciones políticas con posibilidades.

\hypertarget{fragmentaciuxf3n-parlamentaria-2021}{%
\subsection{Fragmentación parlamentaria
2021:}\label{fragmentaciuxf3n-parlamentaria-2021}}

Fparlamentario=1−TantoX1escaños2

\emph{Tabla 1. Elecciones Federales Alemania de 2021. Fragmentación
parlamentaria.}

\begin{longtable}[]{@{}
  >{\raggedright\arraybackslash}p{(\columnwidth - 6\tabcolsep) * \real{0.4000}}
  >{\raggedright\arraybackslash}p{(\columnwidth - 6\tabcolsep) * \real{0.2000}}
  >{\raggedright\arraybackslash}p{(\columnwidth - 6\tabcolsep) * \real{0.2000}}
  >{\raggedright\arraybackslash}p{(\columnwidth - 6\tabcolsep) * \real{0.2000}}@{}}
\toprule\noalign{}
\begin{minipage}[b]{\linewidth}\raggedright
\end{minipage} & \begin{minipage}[b]{\linewidth}\raggedright
\textbf{Escaños 2021}
\end{minipage} & \begin{minipage}[b]{\linewidth}\raggedright
\textbf{TantoX1 escaños}
\end{minipage} & \begin{minipage}[b]{\linewidth}\raggedright
  ( TantoX1 escaños) 2
\end{minipage} \\
\midrule\noalign{}
\endhead
\bottomrule\noalign{}
\endlastfoot
  \textbf{SPD} &   206 &   0,28 &   0,08 \\
  \textbf{CDU/CSU} &   197 &   0,27 &   0,07 \\
  \textbf{VERDES} &   118 &   0,16 &   0,03 \\
  \textbf{FDP ( Liberales )} &   92 &   0,13 &   0,02 \\
  \textbf{AfD (Ultraderecha)} &   83 &   0,11 &   0,01 \\
  \textbf{LA IZQUIERDA} &   39 &   0,05 &   0,03 \\
  \textbf{SSW} &   1 &   0 &   0 \\
   &   Total escaños:736 &    &   Suma: 0,21 \\
   &   \textbf{Fragmentación parlamentaria} &   F = 1- 0,21 =
\textbf{0,79} & \\
\end{longtable}

\hypertarget{operativa-para-el-cuxe1lculo}{%
\subsection{Operativa para el
cálculo}\label{operativa-para-el-cuxe1lculo}}

\begin{enumerate}
\def\labelenumi{\arabic{enumi}.}
\tightlist
\item
  En la 3ª columna ( ``TantoX1 escaños'' ) obtenemos el tanto por uno de
  escaños de cada partido dividiendo el número de escaños que ha
  obtenido por el total de escaños ( 736 ).
\item
  El tanto por uno obtenido se eleva al cuadrado en la 4ª columna.
\item
  Se suman de los valores de la 4ª columna ( ``( TantoX1 escaños )2'') y
  nos da el valor 0,21.
\item
  Se resta este valor a 1.
\end{enumerate}

En nuestro caso, estamos ante un sistema de partidos bastante
fragmentado ( valor de fragmentación más cerca del 1 que del 0 ).

\hypertarget{umbral-electoral}{%
\subsection{Umbral electoral}\label{umbral-electoral}}

La legislación alemana permite acceder al parlamento solo si se supera
la \textbf{barrera legal} del 5\% de los votos, no obstante se exime del
cumplimiento de esta condición a los partidos de minorías étnicas como
es el SSW, partido de la minoría danesa de Schleswig-Holstein. Este
partido solo se enfrenta pues a una \textbf{barrera real}.

Esta consideración con las minorías del sistema electoral alemán
contribuye a un parlamento un poco más fragmentado. El sistema alemán
apuesta por la calidad democrática en este aspecto ( más
representatividad ) frente a la efectividad.

\hypertarget{sistema-de-partidos}{%
\section{4. SISTEMA DE PARTIDOS}\label{sistema-de-partidos}}

Si nos fijamos en los datos de las elecciones del 2017 comparados con
estas últimas ( Tabla 2 ), vemos que en 2017 teníamos un \textbf{sistema
de partidos multipartidista con dos partidos dominantes} según la
clasificación de De Ware. En cambio, ahora, con el crecimiento del
tercer partido ( Los Verdes ) , incluso del cuarto ( el partido liberal
), el sistema se acerca más a un \textbf{multipartidismo segmentado}.

\emph{Tabla2. Comparativa resultados elecciones federales alemanas 2017
y 2021.}

\begin{longtable}[]{@{}
  >{\raggedright\arraybackslash}p{(\columnwidth - 8\tabcolsep) * \real{0.2308}}
  >{\centering\arraybackslash}p{(\columnwidth - 8\tabcolsep) * \real{0.2308}}
  >{\centering\arraybackslash}p{(\columnwidth - 8\tabcolsep) * \real{0.2308}}
  >{\raggedright\arraybackslash}p{(\columnwidth - 8\tabcolsep) * \real{0.1538}}
  >{\raggedright\arraybackslash}p{(\columnwidth - 8\tabcolsep) * \real{0.1538}}@{}}
\toprule\noalign{}
\begin{minipage}[b]{\linewidth}\raggedright
   
\end{minipage} & \begin{minipage}[b]{\linewidth}\centering
   \textbf{2021}
\end{minipage} & \begin{minipage}[b]{\linewidth}\centering
   \textbf{2017}
\end{minipage} & \begin{minipage}[b]{\linewidth}\raggedright
\end{minipage} & \begin{minipage}[b]{\linewidth}\raggedright
\end{minipage} \\
\midrule\noalign{}
\endhead
\bottomrule\noalign{}
\endlastfoot
    &    \textbf{Escaños} &    \textbf{\% votos} &    \textbf{Escaños} &
   \textbf{\% votos} \\
   \textbf{SPD} &    206 &    25,7 &    153 &    20,5 \\
   \textbf{CDU/CSU} &    197 &    24,07 &    246 &    32,9 \\
   \textbf{VERDES} &    118 &    14,75 &    67 &    8,9 \\
   \textbf{FDP (Liberales)} &    92 &    11,46 &    80 &    10,7 \\
   \textbf{AfD (Ultraderecha)} &    83 &    10,34 &    94 &    12,6 \\
   \textbf{LA IZQUIERDA} &    39 &    4,89 &    69 &    9,2 \\
   \textbf{SSW} &    1 &    0,1 &    0 &    0 \\
   \textbf{Total} &    \textbf{736} &     &    \textbf{709} &     \\
\end{longtable}

Si atendemos a la clasificación de Giovanni Sartori no cabe duda que
estamos ante un sistema pluralista. Vayamos pues a valorar si es un
\textbf{pluralismo moderado} o

\hypertarget{pluralismo-polarizado}{%
\subsection{Pluralismo polarizado}\label{pluralismo-polarizado}}

En una primera valoración podemos deducir que el hecho de que
socialdemócratas, liberales y demócrata-cristianos compiten en el centro
tenemos un rasgo inequívoco de pluralismo moderado. Incluso Los Verdes
alemanes se han sumado a esta competencia reubicándose en las posiciones
más centradas ( ecoliberalismo ) que representa su candidata. Por
contra, la presencia de la ultraderecha y la izquierda radical como
fuerzas antisistema es una señal clara de pluralismo polarizado.

Así pues, respecto a las anteriores elecciones del 2017, tenemos una
evolución del ecologismo que se suma a la competición por el centro, el
retroceso de los dos partidos antisistema y el cambio de un sistema de
dos partidos predominantes a un sistema de tres. Esto nos permite
afirmar que el sistema de partidos alemán ha evolucionado hacia un
sistema \textbf{más pluralista aunque menos polarizado} y más moderado.

\hypertarget{anuxe1lisis-del-resultado-competitividad-y-concentraciuxf3n}{%
\section{5. ANÁLISIS DEL RESULTADO: COMPETITIVIDAD Y
CONCENTRACIÓN}\label{anuxe1lisis-del-resultado-competitividad-y-concentraciuxf3n}}

\hypertarget{competitividad}{%
\subsection{Competitividad}\label{competitividad}}

Si comparamos los resultados, tanto en porcentaje de votos como de
escaños, con los obtenidos en 2017, podemos obtener el indicador de
Competitividad que refleja la intensidad de la disputa entre los
principales partidos.

La competitividad es la diferencia entre los dos primeros partidos.
Hablaremos de competitividad electoral si consideramos los porcentajes
de voto o de competitividad parlamentaria si consideramos los
porcentajes de escaños.

\emph{Tabla3. Comparativa de competitividad en la elecciones federales
alemanas 2017 y 2021}

\begin{longtable}[]{@{}
  >{\raggedright\arraybackslash}p{(\columnwidth - 4\tabcolsep) * \real{0.3333}}
  >{\centering\arraybackslash}p{(\columnwidth - 4\tabcolsep) * \real{0.3333}}
  >{\centering\arraybackslash}p{(\columnwidth - 4\tabcolsep) * \real{0.3333}}@{}}
\toprule\noalign{}
\begin{minipage}[b]{\linewidth}\raggedright
   
\end{minipage} & \begin{minipage}[b]{\linewidth}\centering
   \textbf{COMPETITIVIDAD 2021}
\end{minipage} & \begin{minipage}[b]{\linewidth}\centering
   \textbf{COMPETITIVIDAD 2017}
\end{minipage} \\
\midrule\noalign{}
\endhead
\bottomrule\noalign{}
\endlastfoot
   Electoral &    1,67 &    12,50 \\
   Parlamentaria &    1,22 &    13,12 \\
\end{longtable}

La competitividad electoral está por debajo del 5\%, es por tanto muy
alta.

\hypertarget{operativa-del-cuxe1lculo}{%
\subsection{Operativa del cálculo:}\label{operativa-del-cuxe1lculo}}

Para la obtención del valor de competitividad electoral se resta el \%
de votos del primer partido y el segundo.

Para la obtención del valor de competitividad parlamentaria se resta el
\% de diputados del primer partido y del segundo.

\hypertarget{concentraciuxf3n}{%
\section{6. CONCENTRACIÓN}\label{concentraciuxf3n}}

A partir de la suma del \% de votos del primer y segundo partido
obtenemos la concentración electoral que nos indica el peso que tiene el
bipartidismo. De forma análoga si tomamos los porcentajes de escaños
obtendremos la concentración parlamentaria.

\emph{Tabla4. Comparativa de concentración de voto en la elecciones
federales alemanas 2017 y 2021}

\begin{longtable}[]{@{}
  >{\raggedright\arraybackslash}p{(\columnwidth - 4\tabcolsep) * \real{0.3333}}
  >{\centering\arraybackslash}p{(\columnwidth - 4\tabcolsep) * \real{0.3333}}
  >{\centering\arraybackslash}p{(\columnwidth - 4\tabcolsep) * \real{0.3333}}@{}}
\toprule\noalign{}
\begin{minipage}[b]{\linewidth}\raggedright
   
\end{minipage} & \begin{minipage}[b]{\linewidth}\centering
   \textbf{CONCENTRACIÓN 2021}
\end{minipage} & \begin{minipage}[b]{\linewidth}\centering
   \textbf{CONCENTRACIÓN 2017}
\end{minipage} \\
\midrule\noalign{}
\endhead
\bottomrule\noalign{}
\endlastfoot
   Electoral &    49,81\% &    53,50\% \\
   Parlamentaria &    54,76\% &    56,28\% \\
\end{longtable}

La mayor pluralidad de la que hablábamos antes se ve también reflejada
en una menor concentración de votos de los dos principales partidos en
las elecciones del 2021 respecto a las anteriores.

\hypertarget{volatilidad-electoral}{%
\section{7. VOLATILIDAD ELECTORAL}\label{volatilidad-electoral}}

Con la volatilidad se cuantifica el cambio de preferencias partidistas
entre elecciones. Para el cálculo de la volatilidad electoral tomaremos
los porcentajes de votos de la segunda papeleta, lista cerrada.

La suma del valor absoluto de porcentaje de votos que ha variado para
cada partido divido por 2.

VolatilidadElectoral=\%votos2021−\%votos20172

\emph{Tabla 4. Volatilidad en la elecciones federales alemanas 2021
respecto a 2017}

\begin{longtable}[]{@{}
  >{\raggedright\arraybackslash}p{(\columnwidth - 6\tabcolsep) * \real{0.2500}}
  >{\centering\arraybackslash}p{(\columnwidth - 6\tabcolsep) * \real{0.2500}}
  >{\centering\arraybackslash}p{(\columnwidth - 6\tabcolsep) * \real{0.2500}}
  >{\centering\arraybackslash}p{(\columnwidth - 6\tabcolsep) * \real{0.2500}}@{}}
\toprule\noalign{}
\begin{minipage}[b]{\linewidth}\raggedright
  
\end{minipage} & \begin{minipage}[b]{\linewidth}\centering
  \textbf{\% votos 2021}
\end{minipage} & \begin{minipage}[b]{\linewidth}\centering
  \textbf{\% votos 2017}
\end{minipage} & \begin{minipage}[b]{\linewidth}\centering
  \texttt{}\textbf{Diferencia}
\end{minipage} \\
\midrule\noalign{}
\endhead
\bottomrule\noalign{}
\endlastfoot
  \textbf{SPD} &   25,74 &   20,5 &   5,24 \\
  \textbf{CDU/CSU} &   24,07 &   32,9 &   8,93 \\
  \textbf{VERDES} &   14,75 &   8,9 &   5,85 \\
  \textbf{FDP (Liberales)} &   11,46 &   10,7 &   0,76 \\
  \textbf{AfD (Ultraderecha)} &   10,34 &   12,6 &   2,26 \\
  \textbf{LA IZQUIERDA} &   4,89 &   9,2 &   4,31 \\
  \textbf{SSW} &   0,1 &   - &   0,1 \\
  \textbf{Volatilidad electoral} &   \textbf{13,73} & & \\
\end{longtable}

\hypertarget{operativa-de-cuxe1lculo}{%
\subsection{Operativa de cálculo:}\label{operativa-de-cuxe1lculo}}

\begin{enumerate}
\def\labelenumi{\arabic{enumi}.}
\tightlist
\item
  Se obtiene el valor absoluto de la diferencia de \% de votos de cada
  partido en las elecciones consecutivas. \textbf{Volatilidad
  desagregada}.
\item
  Se suman estos valores y se divide por 2 y se obtiene la
  \textbf{volatilidad total o agregada}.
\end{enumerate}

El valor va desde 0\% a 100\%. En el caso extremo de desaparecer todos
los partidos, la suma de volatilidades desagregadas daría 200\% ( 100\%
+ 100; los partidos que desaparecen más lo nuevos ) , que al dividir-la
por 2 daría 100\%.

En el otro extremo tendríamos un resultado electoral idéntico al
anterior: 0\%

En las elecciones federales alemanas del 2021, gran parte de la
volatilidad prácticamente es de tipo \textbf{intrasistémica,} se produce
entre partido ya existentes en el Bundestag desde 2017. No obstante
también tenemos la entrada al parlamento del partido de la minoría
danesa de Schleswig-Holstein, el SSW. Teniendo en cuenta que este
partido no se presentó desde 1961 estamos estamos ante un calor ejemplo
de volatilidad \textbf{extrasistémica}.

Si comparamos la suma de los votos de las opciones de centro hasta la
ultraderecha por un lado y la suma de los votos del bloque de izquierda
y ecologismo, por otro podríamos aventurarnos a afirmar que puede haber
habido una volatilidad \textbf{interbloques} además de la
\textbf{intrabloque.}

\emph{Tabla5. Suma de porcentajes de voto por bloques ideológicos según
clivaje socioeconómico ``Derecha-Izquierda''.}

\begin{longtable}[]{@{}lcc@{}}
\toprule\noalign{}
\textbf{BLOQUE} & \textbf{2021} & \textbf{2017} \\
\midrule\noalign{}
\endhead
\bottomrule\noalign{}
\endlastfoot
SPD+ VERDES + LA IZQUIERDA & 45,38\% & 38,6\% \\
CDU/CSU+ FPD + AfD & 45,87\% & 56,3\% \\
\end{longtable}

\hypertarget{gobernabilidad-y-representatividad}{%
\section{8. GOBERNABILIDAD Y
REPRESENTATIVIDAD}\label{gobernabilidad-y-representatividad}}

Analicemos los factores que favorecen la gobernabilidad y los que la
complican así como su relación con la representatividad, especialmente
después del 2021.

\hypertarget{aspectos-que-complican-la-gobernabilidad}{%
\subsection{Aspectos que complican la
gobernabilidad}\label{aspectos-que-complican-la-gobernabilidad}}

El sistema electoral alemán en sí apuesta claramente por logar una cuota
alta de representatividad de la sociedad en el Bundestag y sacrifica la
eficacia deseable para facilitar gobiernos sin bloqueos. Se trata de un
sistema electoral proporcional, un rasgo coherente con otros propios de
una \textbf{democracia consensual} tanto en la dimensión
\emph{ejecutivos-partidos} ( gobierno de coalición, equilibrio de
poderes, multipartidismo o corporativismo ) como en la dimensión
\emph{federal-unitaria} ( federalismo, bicameralismo\ldots{} ) que tiene
la democracia alemana.

Este hecho se refleja no solo en el resultado electoral fragmentado o en
la oferta plural de partidos sino también en aspectos que hemos
comentado como el posible incremento del número de diputados para buscar
una mayor proporcionalidad o la exención de la barrera legal del 5\% a
opciones que representen minorías étnicas.

\hypertarget{aspectos-que-ayudan-a-la-gobernabilidad.-el-pluralismo-moderado}{%
\subsection{Aspectos que ayudan a la gobernabilidad. El pluralismo
moderado}\label{aspectos-que-ayudan-a-la-gobernabilidad.-el-pluralismo-moderado}}

A pesar de todo, los partidos políticos solucionan de forma responsable
esta situación. Se da la paradoja que esta mayor fragmentación del
sistema producida en 2021, se ha dado de forma paralela a una mayor
competición por el centro político con el giro ecoliberal de Los Verdes
y un descenso de los partidos extremistas. De hecho fue este nuevo
escenario de pluralismo moderado ell que abrió varias posibilidades de
negociación para alcanzar un gobierno de coalición. Recordemos que las
elecciones posibilitan la configuración de gobiernos pero no lo
determinan, esto es tarea de los partidos. Muy pronto la prensa empezó a
hablar de la posible coalición ``Jamaica'' o coalición ``Semáforo'' en
clara alusión de los colores con que identifican los cuatro principales
partidos políticos:

\begin{itemize}
\tightlist
\item
  Coalición Jamaica ( Negro-amarillo-verde ): Demócrata-cristianos,
  liberales y verdes.
\item
  Coalición Semáforo ( Rojo-amarillo-verde ): Socialdemócratas,
  liberales y verdes.
\end{itemize}

Algo destacable es que en ningún todo momento se ha eludido ``toda
alianza con partidos y candidatos antidemocráticos'' ( Levitsky y
Ziblatt: 2018).

\hypertarget{bibliografia-y-webgrafuxeda}{%
\section{9. BIBLIOGRAFIA Y
WEBGRAFÍA}\label{bibliografia-y-webgrafuxeda}}

\begin{itemize}
\item
  Deustcher Bundestag. (s.f ). \emph{Elecciones}.
  \url{https://www.bundestag.de/es/wahlen}
\item
  Isabel Diz Otero, María Lois González y Amparo Novo Vázquez.
  \emph{Ciencia política contemporánea.} 2012. Editorial UOC.
\item
  Steven Levitsky y Daniel Ziblatt, \emph{Cómo mueren las democracias.}
  2018. Editorial Planeta.
\end{itemize}

\end{document}
