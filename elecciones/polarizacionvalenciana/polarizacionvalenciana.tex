% Options for packages loaded elsewhere
\PassOptionsToPackage{unicode}{hyperref}
\PassOptionsToPackage{hyphens}{url}
%
\documentclass[
]{article}
\usepackage{amsmath,amssymb}
\usepackage{iftex}
\ifPDFTeX
  \usepackage[T1]{fontenc}
  \usepackage[utf8]{inputenc}
  \usepackage{textcomp} % provide euro and other symbols
\else % if luatex or xetex
  \usepackage{unicode-math} % this also loads fontspec
  \defaultfontfeatures{Scale=MatchLowercase}
  \defaultfontfeatures[\rmfamily]{Ligatures=TeX,Scale=1}
\fi
\usepackage{lmodern}
\ifPDFTeX\else
  % xetex/luatex font selection
\fi
% Use upquote if available, for straight quotes in verbatim environments
\IfFileExists{upquote.sty}{\usepackage{upquote}}{}
\IfFileExists{microtype.sty}{% use microtype if available
  \usepackage[]{microtype}
  \UseMicrotypeSet[protrusion]{basicmath} % disable protrusion for tt fonts
}{}
\makeatletter
\@ifundefined{KOMAClassName}{% if non-KOMA class
  \IfFileExists{parskip.sty}{%
    \usepackage{parskip}
  }{% else
    \setlength{\parindent}{0pt}
    \setlength{\parskip}{6pt plus 2pt minus 1pt}}
}{% if KOMA class
  \KOMAoptions{parskip=half}}
\makeatother
\usepackage{xcolor}
\usepackage[margin=1in]{geometry}
\usepackage{longtable,booktabs,array}
\usepackage{calc} % for calculating minipage widths
% Correct order of tables after \paragraph or \subparagraph
\usepackage{etoolbox}
\makeatletter
\patchcmd\longtable{\par}{\if@noskipsec\mbox{}\fi\par}{}{}
\makeatother
% Allow footnotes in longtable head/foot
\IfFileExists{footnotehyper.sty}{\usepackage{footnotehyper}}{\usepackage{footnote}}
\makesavenoteenv{longtable}
\usepackage{graphicx}
\makeatletter
\def\maxwidth{\ifdim\Gin@nat@width>\linewidth\linewidth\else\Gin@nat@width\fi}
\def\maxheight{\ifdim\Gin@nat@height>\textheight\textheight\else\Gin@nat@height\fi}
\makeatother
% Scale images if necessary, so that they will not overflow the page
% margins by default, and it is still possible to overwrite the defaults
% using explicit options in \includegraphics[width, height, ...]{}
\setkeys{Gin}{width=\maxwidth,height=\maxheight,keepaspectratio}
% Set default figure placement to htbp
\makeatletter
\def\fps@figure{htbp}
\makeatother
\setlength{\emergencystretch}{3em} % prevent overfull lines
\providecommand{\tightlist}{%
  \setlength{\itemsep}{0pt}\setlength{\parskip}{0pt}}
\setcounter{secnumdepth}{-\maxdimen} % remove section numbering
\ifLuaTeX
  \usepackage{selnolig}  % disable illegal ligatures
\fi
\IfFileExists{bookmark.sty}{\usepackage{bookmark}}{\usepackage{hyperref}}
\IfFileExists{xurl.sty}{\usepackage{xurl}}{} % add URL line breaks if available
\urlstyle{same}
\hypersetup{
  pdftitle={LA POLARIZACIÓN EN LES CORTS VALENCIANES},
  pdfauthor={Tomàs Ferrandis Moscardó},
  hidelinks,
  pdfcreator={LaTeX via pandoc}}

\title{LA POLARIZACIÓN EN LES CORTS VALENCIANES}
\author{Tomàs Ferrandis Moscardó}
\date{2023-12-10}

\begin{document}
\maketitle

{
\setcounter{tocdepth}{2}
\tableofcontents
}
\hypertarget{nepe-y-el-nepp-elecciones-autonuxf3micas-2019-comunidad-valenciana}{%
\section{1 NEPE Y EL NEPP ELECCIONES AUTONÓMICAS 2019 COMUNIDAD
VALENCIANA**}\label{nepe-y-el-nepp-elecciones-autonuxf3micas-2019-comunidad-valenciana}}

\hypertarget{cuxe1lculos}{%
\subsection{1.1 CÁLCULOS}\label{cuxe1lculos}}

Obtenemos los datos de Número Efectivo de Partidos Electorales y Número
Efectivo de Partidos Parlamentarios.

\begin{longtable}[]{@{}lrrrrrr@{}}
\toprule\noalign{}
Sigla & Votos & Candidatura & Escaños & \% escaños & TantoX1escaños &
Tanto1Xotos \\
\midrule\noalign{}
\endhead
\bottomrule\noalign{}
\endlastfoot
PSOE & 643.909 & 24,40 & 27 & 27 & 0,24 & 0,27 \\
PP & 508.534 & 19,27 & 19 & 19 & 0,19 & 0,19 \\
Cs & 470.676 & 17,84 & 18 & 18 & 0,18 & 0,18 \\
COMPROMÍS & 443.640 & 16,81 & 17 & 17 & 0,17 & 0,17 \\
VOX & 281.608 & 10,67 & 10 & 10 & 0,11 & 0,10 \\
PODEMOS & 215.392 & 8,16 & 8 & 8 & 0,08 & 0,08 \\
& & & & Suma & 0,18 & 0,19 \\
& & & & F=1-Σx1² & 0,83 & 0,81 \\
& & & & & \textbf{NEPP} & \textbf{NEPE} \\
& & & & & 5,71 & 5,38 \\
\end{longtable}

\emph{Fuente: Elaboración propia a partir de datos de ARGOS}

\textbf{Sigla:} Siglas con las que concurría el partido o coalición
electoral.

\textbf{Votos:} Suma de los votos de cada candidatura en las tres
circunscripciones provinciales.

\textbf{\%Válidos:} Porcentaje de votos válidos de cada candidatura
redondeado a 2 decimales.

\textbf{Escaños:} Suma de escaños obtenidos por cada candidatura.

\textbf{x1\_votos:} Tanto por uno de votos de cada candidatura,
redondeado a 3 decimales.

\textbf{x\_1\_escaños:} Tanto por uno respecto al total de escaños
asignados a la autonomía.

\textbf{x1\_votos2:} x1\_votos elevado al cuadrado y redondeado a 3
decimales

\textbf{x1\_escaños2:} x1\_escaños elevado al cuadrado y redondeado a 3
decimales.

\textbf{NEPE:} Número Efectivo de Partidos Electorales.

NEPE=1TantoX1votos2

\textbf{Número efectivo de partidos parlamentarios} \textbf{NEPP:}
Número Efectivo de Partidos Parlamentarios.

NEPP=1TantoX1escaños2

\hypertarget{la-diferencia-entre-el-nepe-y-el-nepp}{%
\section{1.2 LA DIFERENCIA ENTRE EL NEPE Y EL
NEPP}\label{la-diferencia-entre-el-nepe-y-el-nepp}}

La razón por la que el NEPP es mayor que el NEPE es la
desproporcionalidad que introduce el sistema electoral a la hora de
transformar los votos en escaños.

Esta desproporcionalidad se debe a dos efectos. Por un parte, la
asignación de diputados aplicando la \textbf{Ley d'Hondt}. Esta fórmula
favorece al partido o los partidos mayoritarios en detrimento del resto.
Por otra parte, tenemos el efecto producido por la división del
territorio en \textbf{circunscripciones provinciales.} Al crear estos
distritos, se asignan menos escaños en cada reparto, hecho que dificulta
(cuando no impide) a los terceros partidos la obtención de
representantes.

La consecuencia de ambos efectos es una concentración de escaños
superior a la de votos. \#\#\# \textbf{EFECTO DE LA APLICACIÓN DE LA LEY
D'HONDT} \#\#\#\# \textbf{\emph{Circunscripción de valencia}}

\begin{longtable}[]{@{}
  >{\raggedright\arraybackslash}p{(\columnwidth - 12\tabcolsep) * \real{0.1000}}
  >{\centering\arraybackslash}p{(\columnwidth - 12\tabcolsep) * \real{0.1500}}
  >{\centering\arraybackslash}p{(\columnwidth - 12\tabcolsep) * \real{0.1500}}
  >{\centering\arraybackslash}p{(\columnwidth - 12\tabcolsep) * \real{0.1500}}
  >{\centering\arraybackslash}p{(\columnwidth - 12\tabcolsep) * \real{0.1500}}
  >{\centering\arraybackslash}p{(\columnwidth - 12\tabcolsep) * \real{0.1500}}
  >{\centering\arraybackslash}p{(\columnwidth - 12\tabcolsep) * \real{0.1500}}@{}}
\toprule\noalign{}
\begin{minipage}[b]{\linewidth}\raggedright
;;
\end{minipage} & \begin{minipage}[b]{\linewidth}\centering
\textbf{PSOE}
\end{minipage} & \begin{minipage}[b]{\linewidth}\centering
\textbf{PP}
\end{minipage} & \begin{minipage}[b]{\linewidth}\centering
\textbf{VOX}
\end{minipage} & \begin{minipage}[b]{\linewidth}\centering
\textbf{UP}
\end{minipage} & \begin{minipage}[b]{\linewidth}\centering
\textbf{COMPROMÍS}
\end{minipage} & \begin{minipage}[b]{\linewidth}\centering
\textbf{Cs}
\end{minipage} \\
\midrule\noalign{}
\endhead
\bottomrule\noalign{}
\endlastfoot
40 & 10 & 7 & 4 & 3 & 9 & 7 \\
1 & \textbf{326.663} & \textbf{259.934} & \textbf{146.908} &
\textbf{107.397} & \textbf{299.404} & \textbf{250.267} \\
2 & \textbf{163.332} & \textbf{129.967} & \textbf{73.454} &
\textbf{53.699} & \textbf{149.702} & \textbf{125.134} \\
3 & \textbf{108.888} & \textbf{86.645} & \textbf{48.969} &
\textbf{35.799} & \textbf{99.801} & \textbf{83.422} \\
4 & \textbf{81.666} & \textbf{64.984} & \textbf{36.727} & 26.849 &
\textbf{74.851} & \textbf{62.567} \\
5 & \textbf{65.333} & \textbf{51.987} & 29.382 & 21.479 &
\textbf{59.881} & \textbf{50.053} \\
6 & \textbf{54.444} & \textbf{43.322} & 24.485 & 17.900 &
\textbf{49.901} & \textbf{41.711} \\
7 & \textbf{46.666} & \textbf{37.133} & 20.987 & 15.342 &
\textbf{42.772} & \textbf{35.752} \\
8 & \textbf{40.833} & 32.492 & 18.364 & 13.425 & \textbf{37.426} &
31.283 \\
9 & \textbf{36.296} & 28.882 & 16.323 & 11.933 & \textbf{33.267} &
27.807 \\
10 & \textbf{32.666} & 25.993 & 14.691 & 10.740 & 29.940 & 25.027 \\
11 & 29.697 & 23.630 & 13.355 & 9.763 & 27.219 & 22.752 \\
\end{longtable}

\emph{Fuente: Elaboración propia a partir de datos de ARGOS}

Vemos como el porcentaje de escaños supera al de votos sólo en los dos
partidos más votados. Siendo el efecto, en el resto de partidos, el
contrario.

\begin{longtable}[]{@{}lcccc@{}}
\toprule\noalign{}
\textbf{Candidaturas} & \textbf{Votos} & \textbf{\% Votos} &
\textbf{Escaños} & \textbf{\% escaños} \\
\midrule\noalign{}
\endhead
\bottomrule\noalign{}
\endlastfoot
P.S.O.E. & 326.663 & 22,8 & 10 & 25 \\
COMPROMíS & 299.404 & 20,9 & 9 & 22,5 \\
PP & 259.934 & 18,15 & 7 & 17,5 \\
Cs & 250.267 & 17,47 & 7 & 17,5 \\
VOX & 146.908 & 10,26 & 4 & 10 \\
UNIDES PODEM-EUPV & 107.397 & 7,5 & 3 & 7,5 \\
\end{longtable}

\emph{Fuente: Elaboración propia a partir de datos de ARGOS}

\hypertarget{section}{%
\paragraph{}\label{section}}

\hypertarget{circunscripciuxf3n-de-alicante}{%
\paragraph{\texorpdfstring{\textbf{\emph{Circunscripción de
Alicante}}}{Circunscripción de Alicante}}\label{circunscripciuxf3n-de-alicante}}

\begin{longtable}[]{@{}
  >{\raggedright\arraybackslash}p{(\columnwidth - 12\tabcolsep) * \real{0.1000}}
  >{\centering\arraybackslash}p{(\columnwidth - 12\tabcolsep) * \real{0.1500}}
  >{\centering\arraybackslash}p{(\columnwidth - 12\tabcolsep) * \real{0.1500}}
  >{\centering\arraybackslash}p{(\columnwidth - 12\tabcolsep) * \real{0.1500}}
  >{\centering\arraybackslash}p{(\columnwidth - 12\tabcolsep) * \real{0.1500}}
  >{\centering\arraybackslash}p{(\columnwidth - 12\tabcolsep) * \real{0.1500}}
  >{\centering\arraybackslash}p{(\columnwidth - 12\tabcolsep) * \real{0.1500}}@{}}
\toprule\noalign{}
\begin{minipage}[b]{\linewidth}\raggedright
\textbf{35}
\end{minipage} & \begin{minipage}[b]{\linewidth}\centering
\textbf{PSOE}
\end{minipage} & \begin{minipage}[b]{\linewidth}\centering
\textbf{PP}
\end{minipage} & \begin{minipage}[b]{\linewidth}\centering
\textbf{VOX}
\end{minipage} & \begin{minipage}[b]{\linewidth}\centering
\textbf{UP}
\end{minipage} & \begin{minipage}[b]{\linewidth}\centering
\textbf{Cs}
\end{minipage} & \begin{minipage}[b]{\linewidth}\centering
\textbf{COMPROMÍS}
\end{minipage} \\
\midrule\noalign{}
\endhead
\bottomrule\noalign{}
\endlastfoot
& 10 & 7 & 4 & 3 & 7 & 4 \\
1 & \textbf{231.662} & \textbf{181.867} & \textbf{102.239} &
\textbf{83.855} & \textbf{173.468} & \textbf{99.488} \\
2 & \textbf{115.831} & \textbf{90.934} & \textbf{51.120} &
\textbf{41.928} & \textbf{86.734} & \textbf{49.744} \\
3 & \textbf{77.221} & \textbf{60.622} & \textbf{34.080} &
\textbf{27.952} & \textbf{57.823} & \textbf{33.163} \\
4 & \textbf{57.916} & \textbf{45.467} & \textbf{25.560} & 20.964 &
\textbf{43.367} & \textbf{24.872} \\
5 & \textbf{46.332} & \textbf{36.373} & 20.448 & 16.771 &
\textbf{34.694} & 19.898 \\
6 & \textbf{38.610} & \textbf{30.311} & 17.040 & 13.976 &
\textbf{28.911} & 16.581 \\
7 & \textbf{33.095} & \textbf{25.981} & 14.606 & 11.979 &
\textbf{24.781} & 14.213 \\
8 & \textbf{28.958} & 22.733 & 12.780 & 10.482 & 21.684 & 12.436 \\
9 & \textbf{25.740} & 20.207 & 11.360 & 9.317 & 19.274 & 11.054 \\
10 & \textbf{23.166} & 18.187 & 10.224 & 8.386 & 17.347 & 9.949 \\
11 & 21.060 & 16.533 & 9.294 & 7.623 & 15.770 & 9.044 \\
\end{longtable}

\emph{Fuente: Elaboración propia a partir de datos de ARGOS}

Vemos cómo claramente el primer partido obtiene un porcentaje de escaños
superior al de votos. El segundo y tercer partido prácticamente se
quedan igual pero el resto quedan perjudicados en el reparto.

\begin{longtable}[]{@{}lcccc@{}}
\toprule\noalign{}
\textbf{Candidaturas} & \textbf{Votos} & \textbf{\% votos} &
\textbf{Escaños} & \textbf{\% Escaños} \\
\midrule\noalign{}
\endhead
\bottomrule\noalign{}
\endlastfoot
PSOE & 231.662 & 25,78 & 10 & 28,57 \\
PP & 181.867 & 20,24 & 7 & 20,00 \\
Cs & 173.468 & 19,3 & 7 & 20,00 \\
VOX & 102.239 & 11,38 & 4 & 11,43 \\
COMPROMíS & 99.488 & 11,07 & 4 & 11,43 \\
UNIDES PODEM-EUPV & 83.855 & 9,33 & 3 & 8,57 \\
\end{longtable}

\emph{Fuente: Elaboración propia a partir de ARGOS}

\hypertarget{circunscripciuxf3n-por-castelluxf3n}{%
\paragraph{\texorpdfstring{\textbf{\emph{Circunscripción por
Castellón}}}{Circunscripción por Castellón}}\label{circunscripciuxf3n-por-castelluxf3n}}

\begin{longtable}[]{@{}lcccccc@{}}
\toprule\noalign{}
& \textbf{PSOE} & \textbf{PP} & \textbf{VOX} & \textbf{UP} & \textbf{Cs}
& \textbf{Compromís} \\
\midrule\noalign{}
\endhead
\bottomrule\noalign{}
\endlastfoot
24 & 7 & 5 & 2 & 2 & 4 & 4 \\
1 & \textbf{85.584} & \textbf{66.733} & \textbf{32.461} &
\textbf{24.140} & \textbf{46.941} & \textbf{44.748} \\
2 & \textbf{42.792} & \textbf{33.367} & \textbf{16.231} &
\textbf{12.070} & \textbf{23.471} & \textbf{22.374} \\
3 & \textbf{28.528} & \textbf{22.244} & 10.820 & 8.047 & \textbf{15.647}
& \textbf{14.916} \\
4 & \textbf{21.396} & \textbf{16.683} & 8.115 & 6.035 & 11.735 &
11.187 \\
5 & \textbf{17.117} & \textbf{13.347} & 6.492 & 4.828 & 9.388 & 8.950 \\
6 & \textbf{14.264} & 11.122 & 5.410 & 4.023 & 7.824 & 7.458 \\
7 & \textbf{12.226} & 9.533 & 4.637 & 3.449 & 6.706 & 6.393 \\
8 & 10.698 & 8.342 & 4.058 & 3.018 & 5.868 & 5.594 \\
\end{longtable}

\emph{Fuente: Elaboración propia a partir de datos de ARGOS}

Aunque con diferencias en los partidos segundo, tercero y cuarto menos
significativas, sigue produciéndose una asignación más favorable al
partido mayor y la asignación más desfavorable a los partidos mas
pequeños.

\begin{longtable}[]{@{}lcccc@{}}
\toprule\noalign{}
\textbf{Candidaturas} & \textbf{Votos} & \textbf{\% votos} &
\textbf{Escaños} & \textbf{\% Escaños} \\
\midrule\noalign{}
\endhead
\bottomrule\noalign{}
\endlastfoot
P.S.O.E & 85.584 & 27,84 & 7 & 29,17 \\
PP & 66.733 & 21,71 & 5 & 20,83 \\
Cs & 46.941 & 15,27 & 4 & 16,67 \\
COMPROMíS & 44.748 & 14,56 & 4 & 16,67 \\
VOX & 32.461 & 10,56 & 2 & 8,33 \\
UNIDES PODEM-EUPV & 24.140 & 7,85 & 2 & 8,33 \\
\end{longtable}

\emph{Fuente: Elaboración propia a partir de datos de ARGOS} \#\# \#\#
\textbf{EFECTO DE LAS CIRCUNSCRIPCIONES PROVINCIALES}

El hecho de que cada escaño no ``cueste'' en todas las circunscripciones
electorales supone una ventaja enorme para los partidos mayoritarios
especialmente en circunscripciones menores. Esto se ve claramente en la
circunscripción de Castellón. Aunque la tendencia a la fragmentación
producida en estas últimas elecciones ha reducido este efecto mucho
respecto a convocatorias anteriores.

\begin{longtable}[]{@{}lccc@{}}
\toprule\noalign{}
& \textbf{Votos} & \textbf{Escaños} & \textbf{Votos por escaños} \\
\midrule\noalign{}
\endhead
\bottomrule\noalign{}
\endlastfoot
Valencia & 1.432.434 & 40 & 35.810 \\
Alicante & 898.706 &  35 &  25.677 \\
 Castellón &  307.396 &  24 &  12.808 \\
\end{longtable}

\emph{Fuente: Elaboración propia a partir de datos de ARGOS}

\hypertarget{section-1}{%
\subsection{**}\label{section-1}}

\hypertarget{conclusiuxf3n}{%
\subsection{\texorpdfstring{\textbf{CONCLUSIÓN}}{CONCLUSIÓN}}\label{conclusiuxf3n}}

De los datos anteriores se puede entender que el ``coste'' de un
Diputado no sea el mismo para cada partido y que, en lineas generales
cuando mayor sea el partido, más fácil sea obtener un diputado más.

\begin{longtable}[]{@{}
  >{\raggedright\arraybackslash}p{(\columnwidth - 6\tabcolsep) * \real{0.1818}}
  >{\centering\arraybackslash}p{(\columnwidth - 6\tabcolsep) * \real{0.2727}}
  >{\centering\arraybackslash}p{(\columnwidth - 6\tabcolsep) * \real{0.2727}}
  >{\centering\arraybackslash}p{(\columnwidth - 6\tabcolsep) * \real{0.2727}}@{}}
\toprule\noalign{}
\begin{minipage}[b]{\linewidth}\raggedright
 
\end{minipage} & \begin{minipage}[b]{\linewidth}\centering
 \textbf{Votos totales}
\end{minipage} & \begin{minipage}[b]{\linewidth}\centering
 \textbf{Diputados}
\end{minipage} & \begin{minipage}[b]{\linewidth}\centering
 \textbf{Votos/Escaño}
\end{minipage} \\
\midrule\noalign{}
\endhead
\bottomrule\noalign{}
\endlastfoot
 PSOE &  643.909 &  27 &  23.848 \\
 PP &  508.534 &  19 &  26.765 \\
 Cs &  470.676 &  18 &  26.149 \\
 COMPROMÍS &  443.640 &  17 &  26.096 \\
 VOX &  281.608 &  10 &  28.161 \\
 PODEMOS &  215.392 &  8 &  26.924 \\
\end{longtable}

\emph{Fuente: Elaboración propia a partir de datos de ARGOS}

\hypertarget{relaciuxf3n-entre-polarizaciuxf3n-y-formato-del-sistema-de-partidos}{%
\section{\texorpdfstring{\textbf{2 RELACIÓN ENTRE POLARIZACIÓN Y FORMATO
DEL SISTEMA DE
PARTIDOS}}{2 RELACIÓN ENTRE POLARIZACIÓN Y FORMATO DEL SISTEMA DE PARTIDOS}}\label{relaciuxf3n-entre-polarizaciuxf3n-y-formato-del-sistema-de-partidos}}

En principio la distribución de fuerzas y su peso electoral y
parlamentario nos sitúa en un escenario claro de multipartidismo.
Respecto a la polarización del sistema podemos señalar aspectos como la
competencia fuerte por el electorado ideologizado de izquierda entre
Compromís y Podemos y la actitud antisistema de VOX (oposición frontal
al autogobierno, al bilingüismo, a la recuperación del derecho civil o
la reforma de la financiación autonómica\ldots). Ambas conductas nos
definen tendencias centrífugas propias de un multipartidismo polarizado.

Aunque, por otra parte, también tendríamos la tendencia centrípeta
propia de un partido \emph{catch-all} como el PSOE y, últimamente,
también recuperada en el PP valenciano que, de hecho, evita hablar de
coalición con la derecha radical así como coincidir en el discurso.

La polarización en si, adquiere una mayor importancia debido a dos
factores. La identidad del País Valenciano como tal y los resultados
electorales similares de las segundas y terceras fuerzas parlamentarias.
Hablamos de un pueblo histórico con una cultura y lengua que no se
ajusta al viejo ideal de Estado-nación uniforme que defiende VOX.
Además, existe una izquierda de marcado acento valencianista. Esto
añadiría una nueva dimensión a la polarización al tener que considerarla
no solo respecto al clivaje típico de izquierda-derecha, sino también
centro-periferia o identitario Además como hemos avanzado, si
consideramos la escasa diferencia de votos y escaños en términos (sobre
todo) de los partidos más polarizados ( VOX, Compromís, Podemos),
incluso un cálculo de la ``polarización ponderada'' nos daría un
resultado elevado tanto electoral como parlamentaria.

No obstante, debemos tener en cuenta que el actual gobierno de coalición
ya es el segundo (Botànic 2) y en estos momentos todos sus componentes
(PSOE Compromís y Podemos) piden el voto para reeditar un Tercer Botànic
frente a la posibilidad, ya real, de gobierno PP-VOX. Incluso el
enfrentamiento entre Compromís y Podemos ha disminuido ante el temor
compartido de que el partido morado no sume suficiente para reeditar el
pacto. Por lo tanto, se empieza a asumir por parte de la izquierda, de
VOX y de Ciudadanos la idea de alternancia de coaliciones.

En conclusión, podemos decir que observando la polarización, entre otras
cuestiones como la fragmentación, concentración etc, el sistema de
partidos valenciano que evolucionó del bipartidismo al multipartidismo
polarizado actual, parece que tienda hacia algo similar a lo que
Caramandi definió como bipolar; una reedición de la coalición de
izquierdas y la posible alternancia con una nueva coalición de derechas
pero no se trataría, de momento, de coaliciones preelectorales aunque no
hay que descartar una futura ``confluencia'' entre Compromís y Podemos.

1

\end{document}
