% Options for packages loaded elsewhere
\PassOptionsToPackage{unicode}{hyperref}
\PassOptionsToPackage{hyphens}{url}
%
\documentclass[
]{article}
\usepackage{amsmath,amssymb}
\usepackage{iftex}
\ifPDFTeX
  \usepackage[T1]{fontenc}
  \usepackage[utf8]{inputenc}
  \usepackage{textcomp} % provide euro and other symbols
\else % if luatex or xetex
  \usepackage{unicode-math} % this also loads fontspec
  \defaultfontfeatures{Scale=MatchLowercase}
  \defaultfontfeatures[\rmfamily]{Ligatures=TeX,Scale=1}
\fi
\usepackage{lmodern}
\ifPDFTeX\else
  % xetex/luatex font selection
\fi
% Use upquote if available, for straight quotes in verbatim environments
\IfFileExists{upquote.sty}{\usepackage{upquote}}{}
\IfFileExists{microtype.sty}{% use microtype if available
  \usepackage[]{microtype}
  \UseMicrotypeSet[protrusion]{basicmath} % disable protrusion for tt fonts
}{}
\makeatletter
\@ifundefined{KOMAClassName}{% if non-KOMA class
  \IfFileExists{parskip.sty}{%
    \usepackage{parskip}
  }{% else
    \setlength{\parindent}{0pt}
    \setlength{\parskip}{6pt plus 2pt minus 1pt}}
}{% if KOMA class
  \KOMAoptions{parskip=half}}
\makeatother
\usepackage{xcolor}
\usepackage[margin=1in]{geometry}
\usepackage{graphicx}
\makeatletter
\def\maxwidth{\ifdim\Gin@nat@width>\linewidth\linewidth\else\Gin@nat@width\fi}
\def\maxheight{\ifdim\Gin@nat@height>\textheight\textheight\else\Gin@nat@height\fi}
\makeatother
% Scale images if necessary, so that they will not overflow the page
% margins by default, and it is still possible to overwrite the defaults
% using explicit options in \includegraphics[width, height, ...]{}
\setkeys{Gin}{width=\maxwidth,height=\maxheight,keepaspectratio}
% Set default figure placement to htbp
\makeatletter
\def\fps@figure{htbp}
\makeatother
\usepackage{soul}
\setlength{\emergencystretch}{3em} % prevent overfull lines
\providecommand{\tightlist}{%
  \setlength{\itemsep}{0pt}\setlength{\parskip}{0pt}}
\setcounter{secnumdepth}{-\maxdimen} % remove section numbering
\ifLuaTeX
  \usepackage{selnolig}  % disable illegal ligatures
\fi
\IfFileExists{bookmark.sty}{\usepackage{bookmark}}{\usepackage{hyperref}}
\IfFileExists{xurl.sty}{\usepackage{xurl}}{} % add URL line breaks if available
\urlstyle{same}
\hypersetup{
  pdftitle={PUNTOS DE VISTA ACTUALES SOBRE EL PAPEL ECONÓMICO DEL ESTADO},
  pdfauthor={Tomàs Ferrandis Moscardó},
  hidelinks,
  pdfcreator={LaTeX via pandoc}}

\title{PUNTOS DE VISTA ACTUALES SOBRE EL PAPEL ECONÓMICO DEL ESTADO}
\author{Tomàs Ferrandis Moscardó}
\date{Octubre de 2022}

\begin{document}
\maketitle

{
\setcounter{tocdepth}{2}
\tableofcontents
}
\hypertarget{introducciuxf3n}{%
\section{1. INTRODUCCIÓN}\label{introducciuxf3n}}

Pronunciarse sobre el papel del Estado en un campo tan complejo como la
economía no es tarea fácil. De hecho, no encontraremos precisamente
unanimidad de opinión al respecto. Unos economistas son detractores de
la intervención del Estado y, a la vista de los fallos del gobierno,
concluyen que puede ser peor el remedio que la enfermedad. Otros, en
cambio ven en los fallos del mercado y en la ausencia de un Estado
fuerte, el germen de las recientes crisis económicas mundiales. Lo más
preocupante, según los pronósticos de estos últimos, es que si no se
adoptan nuevas políticas públicas en favor de un mayor dirigismo o
responsabilidad del Estado para corregir estos fallos tendremos más y
peores crisis.

Para el presente trabajo hemos optado por exponer puntos de vista
diferentes de economistas contemporáneos de renombre. Desde partidarios
de un rol activo del sector público a los contrarios. El prestigioso
economista Joseph E. Stiglitz resume esta dicotomía explicando que hay
economistas que creen en la política monetaria como único instrumento,
pero en cambio otros creen que hace falta más y defienden políticas
fiscales más duras (Stiglitz, 2020: 255).

Podremos observar que los más intervencionistas explican la crisis del
2008 como causada por los fallos del mercado a la vez que ponen en
evidencia la incompleta resolución por parte de los Estados y nos urgen,
ahora, a cambios en la política estatal para evitar nuevas crisis
económicas. Se trata de economistas que reaccionan y plantean soluciones
al ``fallo del mercado'' de forma análoga a como lo hizo Keynes para
afrontar la Gran Depresión. Episodio del siglo XX, por cierto, al que
recurren habitualmente como argumento. Es decir, que en todos ellos hay
una voluntad clara de salvar el sistema económico (y político a la vez)
mediante la recuperación de papel del Estado o, mejor aún, de un nuevo
papel reforzado.

Por contra, los economistas nostálgicos del \emph{laissez-faire,} suelen
observar la Historia con una visión amplia y difusa. Vuelven a los
orígenes del liberalismo con la retórica del ``innovismo'' milagroso y
el individualismo libre para concluir con un balance final siempre
positivo donde todo se lo debemos al mercado libre de intervención
pública y todo se soluciona con más mercado y menos intervención
pública.

\hypertarget{el-papel-econuxf3mico-del-estado}{%
\section{2. EL PAPEL ECONÓMICO DEL
ESTADO}\label{el-papel-econuxf3mico-del-estado}}

En los siguientes apartados veremos la visión de economistas actuales
sobre cuál debe ser (o cual no debe ser) el papel del Estado en la
economía. Entendido este como el conjunto de políticas gubernamentales,
decisiones del sector público en general o de los mismos bancos
centrales.

\hypertarget{por-quuxe9-elegimos-a-estos-economistas}{%
\subsubsection{¿Por qué elegimos a estos
economistas?}\label{por-quuxe9-elegimos-a-estos-economistas}}

Algo en lo que todos estaremos de acuerdo es que nos acordamos más del
papel del Estado como ciudadanos o interesados en la politología cuando
la economía no va bien. En plena bonanza económica un libro sobre el
despilfarro público no tiene el mismo éxito que en plena recesión. Ni
siquiera la propuesta de una renta mínima garantizada despierta el mismo
interés entre la ciudadanía. Quizás por la misma razón, los dos
economistas elegidos en primer lugar, Piketty y Mazzucato, son autores
cuyos trabajos se han centrado o desarrollado a partir del estudio de
las crisis económicas recientes. Nos hablarán de lo que más intensamente
hemos vivido en economía: la crisis del 2008 y la derivada de la
pandemia del Covid-19.

Respecto a los otros economistas elegidos y que defienden un enfoque
distinto sobre la función del Estado, su elección obedece a razones
distintas. En primer lugar, McCloskey aporta una visión netamente
liberal no solo en los términos económicos. Profesora, economista e
historiadora, en sus numerosos artículos nunca escatima en críticas
contra la paranoia militarista de la derecha\textbf{,} contra la guerra
de la droga o a favor de todos los derechos de las personas, comenzando
por el de emigrar. Esto le da una pátina de coherencia a su discurso
liberal\footnote{1 McCloskey además de feminista es defensora de todas
  las libertades civiles, ha publicado artículos en defensa de la
  despenalización del consumo de drogas y la reducción de gasto en
  seguridad.}. Pero sobre todo es su manifiesta oposición a la visión
socialdemócrata de Piketty así como las críticas a Mazzucato lo que
obliga a incluirla en este trabajo.

Aparecen referencias a otros dos economistas contemporáneos de ideas
claramente distintas. El liberal clásico, Guy Sorman, francés y
norteamericano, nos aporta una interesante comparación sobre la visión
tan distinta del Estado y del mercado que tienen en EEUU y en Europa con
la excepción que siempre apunta del Reino Unido; algo que puede
ayudarnos un poco a entender otra crisis reciente, la del Brexit. Si
Sorman participó activamente en la política francesa junto al presidente
Chirac, el Nobel\footnote{1 Premio de Ciencias Económicas del Banco de
  Suecia en Memoria de Alfred Nobel.} Stiglitz, fue asesor del
presidente Clinton, además de presidente senior del Banco Mundial. De él
se suele decir que fue el economista más citado en 2008 y suyo es el
apelativo de ``fundamentalistas del libre mercado'' en sus críticas al
neoliberalismo y la globalización.

De ambos, Stiglitz y Sorman, elegimos algunas opiniones que nos sirven
para reforzar, contradecir o establecer relaciones entre las expuestas
por los demás. No en vano ambos han participado de la política económica
de Estado asesorando y gestionando en distintos niveles y esferas del
poder. Son de todos los citados anteriormente, los únicos que han sido
protagonistas del intervencionismo o no-intervencionismo del Estado, por
lo tanto a ellos les asignamos la tarea de reforzar o refutar las
opiniones de los demás.

\hypertarget{estructura-del-trabajo}{%
\subsection{Estructura del trabajo}\label{estructura-del-trabajo}}

A partir de los trabajos de Pikketty ``El capital en el siglo XXI''
haremos un repaso de los peligros que acechan y las dificultades que
atraviesa el Estado del bienestar en su tarea fundamental de asegurar
aquellos derechos que los ilustrados empezaron a reivindicar y las
conquistas sociales posteriormente ganadas.

Además, veremos las propuestas del economista francés ante dos
problemas: el peligro que se cierne sobre el Estado del bienestar y la
posibilidad de nuevas crisis.

Este apartado, que será el de mayor dimensión, se complementará con la
aportación del ensayo ``No desaprovechemos esta crisis'' de Mazzucato
con sus críticas y sus propuestas en una línea similar de salvar el
sistema económico reforzando y a la vez renovando, el papel del Estado.

Como contrapunto, resumimos los argumentos principales de dos
economistas liberales contrarios a la intervención del sector público en
la economía a partir de los artículos periodísticos de McCloskey
recogidos en ``¿Por qué el liberalismo funciona?'' y el ensayo de Guy
Sorman ``La economía no miente''. También a lo largo del texto
encontraremos alguna referencia a ``El capitalismo progresista'' de
Joseph E. Stiglitz servirán como aclaración de las tesis expuestas por
otros autores.

\hypertarget{pero-quuxe9-vamos-a-considerar-como-papel-del-estado}{%
\subsection{Pero ¿qué vamos a considerar como papel del
Estado?}\label{pero-quuxe9-vamos-a-considerar-como-papel-del-estado}}

En adelante, entenderemos el papel o la intervención del Estado en la
economía como el conjunto de tres grandes funciones. Por una parte,
consideremos la acción recaudatoria vinculada a la prestación de
servicios propios de las actuales democracias en tanto que Estados
sociales. Por otra parte, tendríamos la función reguladora y de control
de toda la actividad económica y, en tercer lugar, asumiremos el papel
del Estado como productor o poseedor de capital.

\hypertarget{piketty-el-estado-social-para-el-siglo-xxi}{%
\section{3. PIKETTY, EL ESTADO SOCIAL PARA EL SIGLO
XXI}\label{piketty-el-estado-social-para-el-siglo-xxi}}

Tras analizar las desigualdades sociales desde el siglo XVIII, Thomas
Piketty concluye que fueron las grandes guerras las que transformaron la
antigua estructura de desigualdades. Pero al contrario de lo que cabía
esperar, en el siglo XXI, las desigualdades patrimoniales han aumentado
hasta un nivel intolerable.

\hypertarget{razones-para-la-intervenciuxf3n-del-estado}{%
\subsection{Razones para la intervención del
Estado}\label{razones-para-la-intervenciuxf3n-del-estado}}

Al comparar la crisis de 2007-2008 con la Gran Depresión vemos que la
padecida en este siglo pasado tuvo consecuencias menos dramáticas. El
desplome de la producción no superó el 5\% en gran parte de Occidente
mientras que en los años treinta del siglo pasado el desplome y cierre
de empresas fue un hecho generalizado, además hubo un desarrollo
posterior en los países emergentes. Dos grandes diferencias entre lo que
pasaría a llamarse la Gran Recesión de 2008 y la Gran Depresión de los
años 1930 del siglo XX.

La razón de este menor impacto está precisamente en el papel que jugaron
los gobiernos y los bancos centrales ante la crisis. Durante la Gran
Depresión el presidente Hoover siguió a rajatabla la máxima de
``liquidar'' cualquier empresa débil. En cambio, en la crisis de 2008,
hubo una respuesta del Estado inmediata: se intervino.

La política financiera de los gobiernos y la política monetaria de los
bancos centrales convirtió a los Estados ricos en prestamistas de última
hora ante una situación de pánico, se pudo salvar al sistema financiero.
Los Estados se implicaron creando una liquidez que mantuviese vivo al
sistema bancario. Imperó a nivel mundial un pragmatismo y, sobre todo,
se actuó.

Ahora bien, esta actuación, según el economista francés, no es una
solución al problema estructural que originó la crisis. Una crisis
incubada en la opacidad financiera y alimentada con un aumento anterior
de la desigualdad. De hecho, Piketty prevé que vuelvan más crisis
similares al no haberse abordado de raíz estos problemas.

\hypertarget{salvar-el-capitalismo-desde-el-estado-de-nuevo}{%
\subsection{Salvar el capitalismo desde el Estado, de
nuevo}\label{salvar-el-capitalismo-desde-el-estado-de-nuevo}}

A partir del fin de la II Guerra Mundial los Estados empiezan a tener
más vida económica y social, lo que trae paralelamente un debate sobre
este protagonismo que se acentúa en los años 1970 y 1980 y,
posteriormente desde los EEUU\footnote{1 Especialmente por el Tea Party,
  un movimiento político conservador y liberal libertario.}.

Ya entrados en el siglo XXI, tras la crisis de 2008, aparece un nuevo
movimiento que acusa de todos los males al mercado. Tenemos pues dos
puntos de vista irreconciliables que podrían escenificarse en la batalla
que Grecia tuvo que lidiar en el seno de la UE: los \emph{antimercado} y
los \emph{antiestado.}

Piketty reconoce parte de razón de ambos planteamientos y propone
inventar nuevas herramientas para retomar el control de un capitalismo
financiero anárquico, renovando absolutamente los sistemas de impuestos
y gastos, que son ``el corazón del Estado social moderno'' hoy en día
tan ininteligibles que solo pueden generar un peligroso rechazo.

Esa idea de renovar el papel del Estado con un nuevo sistema de
impuestos y gastos para salvar la eficacia, a la vez que el control de
sistema financiero. Ese posicionamiento entre extremismos,
\emph{antiestado} y \emph{antimercado}, que pretende salvar el
capitalismo desde la iniciativa Estatal guarda cierto paralelismo con la
figura de Keynes.

\hypertarget{el-regreso-al-estado}{%
\subsection{El regreso al Estado}\label{el-regreso-al-estado}}

Piketty aboga por el regreso del Estado incidiendo en la función de
recaudación vinculada a la prestación de servicios, así como también en
la necesidad de una reglamentación y un arbitraje transparente más que
en el papel del Estado como productor o poseedor de capital.

Es evidente, y puede usarse como argumento contrario a un ``regreso del
Estado'', que las contribuciones obligatorias se han incrementado de
forma incesante desde 1870. El Estado liberal o de derecho que apenas
tenía que afrontar el gasto militar y una tarea de poco más que asegurar
el orden y garantizar el cumplimiento de la ley nada tiene que ver con
el desarrollo posterior a la II Guerra Mundial del Estado social con su
sistema sanitario público, educación, prestaciones por desempleo y
jubilación entre otras. Este aumento de presión fiscal obedece al
aumento de responsabilidades del Estado con la ciudadanía.

La mejora de prestaciones cualitativamente y cuantitativamente tiene su
plasmación en el presupuesto y supone un reto para el sector público que
aumenta con el paso de los años. El reto empieza por el envejecimiento
de la población y su repercusión en el sistema de pensiones y las
políticas sociales, algo que se agrava con el aumento de la pensión
media.

A o que hay que sumar la necesidad de un sistema educativo que
evolucione a la par que la complejidad de la sociedad y un sistema
sanitario más sofisticado cada año. Con todo, la simple estabilidad
presupuestaria en el sector público ya supone toda una proeza.

Respecto a la fijación de reglas y la observancia de éstas, en cambio,
podemos advertir que la regulación de los mercados financieros desde
1950 a 1970 era más estricta que desde 1980 en adelante (Piketty, 2014).
Esta evolución no puede pasar desapercibida tras la crisis del 2008. La
globalización y, previamente, el triunfo del \emph{tatcherismo} ha
erosionado el rol regulador del Estado.

Además, el Estado, en su papel de productor o poseedor de capital, ha
perdido prácticamente todo el peso con las privatizaciones realizadas
desde finales del siglo XX especialmente del sector industrial y
financiero. Mazzucato, de quien hablaremos más tarde, llega a señalar la
paradoja irónica de que el sector público ha llegado a externalizar
tanto que no es capaz de llevar adelante los propios procedimientos o
contratos de externalizaciones (Mazzucato, 2021).

En el contexto español, los ingresos por variación de activos
financieros empezaron en la década de 1980 bajo el gobierno
socialdemócrata de Felipe González y se incrementaron con el cambio de
gobierno y la entrada del PP con José María Aznar (Vergés, 1999). Miles
de millones de pesetas, después euros, que engrosaron el Capítulo VIII
de los Presupuestos Generales del Estado de esos años para hacer frente
al déficit y la deuda. Para los gobiernos de Felipe González estas
decisiones se debían a exigencias de Europa y se maquillaban como
políticas modernizadoras. En el caso del presidente Aznar y los
gobiernos del PP, eran puntos de su programa electoral y respondían más
a la ideología. Estas decisiones de política económica de Estado, pese a
su importancia, no generaron demasiada polémica ni debate social. Cabe
resaltar que, posteriormente, cuando el gobierno de Zapatero (PSOE)
privatizó alguna empresa si hubo polémica a nivel social: las redes
sociales y el panorama de la prensa había cambiado.

\hypertarget{la-redistribuciuxf3n-moderna}{%
\subsection{La redistribución
moderna}\label{la-redistribuciuxf3n-moderna}}

Cuando hablamos de redistribución en nuestro Estado de bienestar no nos
referimos a una redistribución de riqueza en un sentido literal. No se
trata de ``quitar a los que más tienen para dar a los pobres''. La
redistribución se entiende por un mecanismo que asegura la igualdad a la
hora de acceder a bienes y servicios fundamentales. Una igualdad
entendida en términos de derechos reales, la implementación de los
derechos recogidos en declaraciones o preámbulos de Constituciones,
mediante leyes y servicios contemplados los presupuestos públicos. En el
terreno teórico sobre los derechos para acceder a la felicidad existe un
absoluto consenso formal. El debate se presta cuando se trata de poner
en práctica esa vieja retórica de la Declaración de Derechos del Hombre
y Ciudadano esta es la cara verdadera del debate sobre el papel del
Estado en la economía; la financiación, la presión fiscal.

Así podemos decir que la redistribución actual consiste en financiar los
servicios públicos elementales, jubilaciones, prestaciones por desempleo
y otros subsidios y ayudas similares para que los que más desaventajados
y desaventajadas (no olvidemos la brecha salarial aún existente entre
mujeres y hombres).

Piketty alerta sobre la necesidad de buscar sin cesar la eficacia del
sector público. Centrarse en mejorar el funcionamiento de centros
sanitarios o educativos; el cálculo de las pensiones o subsidios por
desempleo etc. si no queremos que el consenso político sobre el actual
nivel de contribución se rompa y el Estado social entre en peligro.

\hypertarget{impuestos-y-progresividad-fiscal}{%
\subsection{Impuestos y progresividad
fiscal}\label{impuestos-y-progresividad-fiscal}}

Piketty nos recuerda que ``El impuesto no es un asunto técnico, sino
eminentemente político y filosófico, sin duda el primero de todos''
(Piketty, 2014), algo que se demuestra si, por ejemplo, repasamos el
protagonismo que los impuestos tuvieron como detonantes en las
Revolución Francesa\footnote{1 La supresión del feudalismo se
  caracterizó sobre todo por la implantación de la igualdad fiscal en
  1789 el fin de los privilegios en materia fiscal.} y la Revolución
Americana\footnote{2 La decisiva ``not taxation without representation''
  de las colonias americanas con la que empezó la rebelión contra la
  Metropoli al negarse a pagar impuestos aprobados por el parlamento
  inglés.}

Impuestos tenemos típicamente de tres tipos. Los que afectarían al
ingreso, los que afectarían al capital y los que afectarían al consumo.
En España serían, por definición, el IRPF, Impuesto de Sociedades; el
Impuesto sobre el patrimonio y, finalmente, el IVA junto a los impuestos
especiales.

Por otra parte, tendríamos las cotizaciones a la seguridad social para
hacer frente a las jubilaciones y prestaciones por desempleo sin entrar
en las diferencias entre países sobre prestaciones o
financiación.\footnote{1 En algunos países las cotizaciones sociales
  cubren más gastos sociales como Francia. En otros como los países
  nórdicos se cubre parte del gasto social con el impuesto sobre la
  renta.}

El criterio alrededor del cual suele pivotar el debate entre liberales y
socialdemócratas es este tema tan crucial es sobre la progresividad o,
por el contrario, la proporcionalidad de cada impuesto. Si aplicamos la
misma tasa al ingreso, capital o consumo para todos, hablaremos de un
impuesto proporcional. Si, en cambio, la tasa varía en función del
ingreso, capital o consumo, siendo directamente proporcional o por
tramos, estaremos ante un impuesto progresivo.

\hypertarget{la-amenaza-de-la-competencia-fiscal-y-la-regresividad-fiscal}{%
\subsection{La amenaza de la competencia fiscal y la regresividad
fiscal}\label{la-amenaza-de-la-competencia-fiscal-y-la-regresividad-fiscal}}

Cuando afrontamos el impuesto al capital desde la óptica de la
progresividad, la principal amenaza es la llamada competencia fiscal en
el contexto actual de libre circulación de capital. Es decir, la
aplicación de normas fiscales que llevan a cabo países o regiones
destinadas a captar inversiones y actividades empresariales de fuera de
sus fronteras o límites geográficos en el marco de la globalización.

Según explica Piketty, asistimos a una carrera acelerada hacia el
precipicio reduciendo impuestos sobre ingresos a las empresas, pero
manteniendo, a la par, la presión sobre los ingresos por el trabajo.
Este último es el que aporta un mínimo de progresividad en la tasa
impositiva global. Una progresividad que no llega a contrarrestar el
efecto de la proporcionalidad que causa el peso mayor del impuesto al
consumo y las cotizaciones sociales. Con lo cual, gran parte de la tasa
impositiva global la asumen las capas sociales más débiles: podemos
hablar pues de impuestos regresivos.

El riesgo de esta regresividad fiscal no es solo el de acelerar la
desigualdad social y retroceder hacia la concentración del capital en
términos similares o superiores a antes de 1945 sino que además puede
crear una reacción de rechazo del sistema fiscal actual por parte de las
clases medias.

Como afirma Piketty ``es vital para el Estado social moderno que el
sistema social que lo sostiene conserve un mínimo de progresividad, o
cuando menos que no llegue a ser claramente regresivos en la cima''.

Los impuestos progresivos son el pilar del Estado social. Fueron la base
para su construcción el siglo XX y son indispensables para su
mantenimiento en el siglo XXI. Entonces, ¿cómo resolvemos el problema
que nos plantea la competencia fiscal?

\hypertarget{impuesto-mundial-sobre-el-capital.-la-utopuxeda-de-piketty}{%
\subsection{Impuesto mundial sobre el capital. La utopía de
Piketty}\label{impuesto-mundial-sobre-el-capital.-la-utopuxeda-de-piketty}}

Para responder a la pregunta anterior y, así, poder lanzar una propuesta
de ``reactualización del programa social demócrata y fiscal liberal del
siglo pasado'' deberá hacer falta algún instrumento nuevo, sin duda.

De igual forma que en el siglo XX se introdujeron como novedades en la
política fiscal de los Estados, el impuesto progresivo sobre el ingreso
y el impuesto progresivo a las sucesiones y fueron las herramientas que
permitieron las políticas del Estado social, ahora, ante la amenaza de
la competencia fiscal, Piketty propone un nuevo impuesto mundial que
podría compatibilizarse o directamente o progresivamente sustituir los
nacionales.

El primer objetivo del nuevo impuesto sería regular de forma eficaz
cualquier crisis bancaria o financiera gracias a proporcionar una
información y transparencia financiera internacional. Recordemos que la
opacidad del sistema financiero está en la raíz de la crisis del 2008 y
puede repetirse.

El segundo objetivo sería la redistribución justa dentro de los países
de la riqueza. Hay que tener en cuenta que la mayoría de los países ya
gravan el capital, pero sólo el patrimonio inmobiliario y, por norma
general sin la posibilidad de deducir el valor del préstamo del valor
del bien. Los activos de este nuevo impuesto serían tanto inmobiliarios,
financieros como empresariales.

El tercer y más importante de los objetivos sería el de regular el
capitalismo frenando la deriva hacia cuotas cada vez mayores de
desigualdad y financiando el Estado social.

Estaríamos hablando de un nuevo impuesto progresivo sobre el capital a
nivel mundial que permitiría una redistribución de la riqueza, generaría
transparencia democrática y financiera sobre las fortunas; algo
imprescindible si queremos que el sistema financiero y sus flujos a
nivel internacional estén regulados.

\hypertarget{conclusiuxf3n}{%
\subsection{Conclusión}\label{conclusiuxf3n}}

La propuesta de Piketty supondría una actualización o revisión del papel
económico del Estado en su doble función reguladora de la economía por
una parte y la de prestar servicios a la ciudadanía. Una propuesta para
salvar, de nuevo, el capitalismo.

\hypertarget{mazzucato-la-lecciuxf3n-del-covid-19}{%
\section{4. MAZZUCATO, LA LECCIÓN DEL
COVID-19}\label{mazzucato-la-lecciuxf3n-del-covid-19}}

Mazzucato reclama un Estado reforzado en sus tres grandes objetivos: el
de prestar servicios y la necesaria provisión presupuestaria, el de
vigilar que el interés general rija en todas la reglamentación y
acuerdos, sobre todo si participa el sector público, a la vez que
reivindica sin tapujos el papel del Estado como productor o poseedor de
capital.

\hypertarget{la-crisis-del-covid-19}{%
\subsection{La crisis del Covid-19}\label{la-crisis-del-covid-19}}

La crisis del Covid-19 ha funcionado como una radiografía perfecta sobre
la estructura de nuestra economía revelando el papel teórico, genuino si
se quiere, del Estado en contra del ``mandato'' del mercado impuesto
desde la década de los años 1980. De la radiografía, la profesora de
economía italiano-estadounidense Mariana Mazzucato señala tres manchas
extrañas en nuestro sistema.

La primera de ellas afecta a la sanidad pública. Aquellos países ricos
con un sistema sanitario público debilitado por las privatizaciones
anteriores al Covid-19 han salido peor pagados: sus ciudadanos han
sufrido las consecuencias de la enfermedad y la muerte sin atención.

La segunda mancha obedece al error en las asociaciones público-privadas.
La excesiva y relajada confianza en las empresas privadas cuando se
trataba de buscar el bien común ha dejado al Estado sin recursos frente
a la calamidad.

No podemos dejar pasar por alto esta triste oportunidad que nos brinda
la crisis triple (sanitaria, económica y ecológica) para ``repensar para
qué sirven los gobiernos'' y ``hacer un capitalismo distinto''.

La economista propone una solución que, a diferencia de aquel New Deal
al que suele recurrir como ejemplo, no se limita a la intervención del
Estado en determinados sectores de la economía. Mazzucato lanza su idea
de enfocarlo todo a lo que ella denomina ``misiones'': grandes objetivos
a cumplir a los que debe dirigirse principalmente la actividad
económica. Una idea que suma a la reivindicación por la recuperación del
papel inversor del Estado y la utilidad social de las empresas públicas
si caer en el estatismo socialista.

\hypertarget{errores-arrastrados-desde-2008}{%
\subsection{Errores arrastrados desde
2008}\label{errores-arrastrados-desde-2008}}

Al igual que Piketty, también la economista americana señala errores
graves en la política pública desde la crisis del 2008. Lo que podríamos
ya se ha señalado como una mala solución. Estos errores, con la crisis
del Covid-19 se han acentuado complicando enormemente la tarea de los
gobiernos y de todo el sector público a la hora de afrontar la crisis.

De forma esquemática podríamos citar como errores los siguientes:

\begin{itemize}
\item
  Los gobiernos conceden créditos a empresa privadas cuando la deuda
  privada es ya demasiado elevada.
\item
  El sector privado estaba excesivamente ``financiarado''. Muchas
  corporaciones prefirieron por ello recompensar a los accionistas
  mediante la compra de acciones antes que reinvertir pensando en el
  futuro: investigación, formación o bienestar para los empleados.
\item
  La austeridad en los gobiernos, sobre todo desde 2008, como solución a
  la crisis ha debilitado a los servicios públicos sanitarios y
  sociales.
\item
  En los rescates que se hicieron de empresas y de la banca no se
  condicionaron en absoluto en beneficio de persona o del medioambiente.
\item
  Tampoco se han dirigido bien las asociaciones público-privadas
  olvidando el interés general en la orientación de los objetivos ni en
  la inversión o redistribución posterior de los beneficios.
\end{itemize}

A partir de esta mirada crítica, Mazzucato, nos insta a aprender la
lección del 2008 y realiza sus propuestas ante la situación actual.

\hypertarget{las-ayudas-a-las-empresas-deben-estar-condicionadas}{%
\subsection{Las Ayudas a las empresas deben estar
condicionadas}\label{las-ayudas-a-las-empresas-deben-estar-condicionadas}}

En la era post COVID el Estado ha recuperado un protagonismo primigenio
al volver a ser el inversor de primer recurso. Desde transferencias
directas, exenciones y bonificaciones fiscales, créditos con buenas
condiciones avalados por el Estado y una política expansiva de los
Bancos Centrales.

Estas ayudas deberían estar condicionadas, a diferencia de lo que se
hizo en la anterior crisis, al mantenimiento de compromisos de las
empresas con los trabajadores y también el medio ambiente para hacer
frente a la crisis ecológica. El mantenimiento de plantilla por ejemplo
ha sido una de las condiciones que en algunos países esta vez sí han
impuesto\footnote{1 Alemania ha exigido el mantenimiento de los empleos
  y Dinamarca que las empresas no estén domiciliadas en paraísos
  fiscales.}. Algunos de los sectores beneficiados por las ayudas son
estratégicamente muy importantes para conseguir objetivos como los
plasmados en el Acuerdo de París (automóvil o la aviación comercial)
para la reducción de emisiones de CO2.

Mazzucato defiende estos condicionamientos como la oportunidad que no
deberíamos desaprovechar para orientar la economía y corregir los
errores del pasado. Entiende que no se trata de ``dirigismo'', sino de
visión a largo plazo: de futuro.

El caso de la investigación de las vacunas en el Covid-19 ha puesto de
relieve la enorme cuantía de millones que los laboratorios han ingresado
de las arcas públicas en todos los países, especialmente EEUU a lo largo
de décadas para investigación en vacunas, sobretodo.

Pero no únicamente es en el sector de laboratorios farmacéuticos, donde
se ha financiado la investigación con recursos públicos por eso
Mazzucato por este motivo defiende que además los rescates desde el
momento en que se hacen con dinero público deberían condicionarse al
reparto futuro de dividendos entre los ciudadanos.

\hypertarget{el-inversor-de-primer-recurso}{%
\subsection{El inversor de primer
recurso}\label{el-inversor-de-primer-recurso}}

El corolario liberal que asumieron a partir de la segunda mitad del
siglo XX las democracias occidentales respecto al papel inversor del
Estado, Mazzucato lo resume en tres puntos:

\begin{itemize}
\item
  La inversión pública es un despilfarro.
\item
  La economía de mercado tiende al pleno empleo.
\item
  La inversión pública solo debe aparecer cuando haya que corregir un
  fallo de mercado.
\end{itemize}

A lo cual la economista se opone defendiendo que la ``capacidad del
Estado es fruto de la inversión paciente en el seno de las instituciones
públicas, no del dinero lanzado a la economía desde un helicóptero en
tiempo de crisis''.

La crisis del 2008 evidenció la debilidad del Estado y la del Covid-19
lo ha vuelto a hacer. Para Mazzucato ``Desde la crisis del 2008
financiera, las economías han operado muy por debajo de su plena
capacidad''. La política monetaria expansiva sirvió para poner dinero en
manos de quienes ya tenían activos y ningún interés en gastarlo en
actividades vinculadas a la economía real.

``Ahora es necesaria una nueva era de inversión pública para conformar
de nuevo nuestro paisaje tecnológico, productivo y social''. Lo
importante no es tanto el crecimiento de la economía sino la dirección
en la que crece. Esto implica necesariamente que hay que volver a poner
alguien al timón del Estado.

\hypertarget{el-programa-de-empleo-puxfablico}{%
\subsection{El Programa de Empleo
Público}\label{el-programa-de-empleo-puxfablico}}

El pleno empleo es un bien público. Destinar el máximo posible de
recursos en conseguirlo supone tener más consumidores y usuarios con
posibilidad de gastar y activar la economía lo contrario que ocurre
cuando el número de desempleados aumenta.

La propuesta de Mazzucato del Programa de Empleo Público tiene que ser
adaptada a la economía del país como lo era el programa de empleo puesto
en marcha por Rooselvet y tendrá cuatro fines:

\begin{itemize}
\item
  Crear un stock laboral que dependerá del ciclo económico
\item
  Sustituir la solución cortoplacista e improductiva del subsidio por
  desempleo por una empleabilidad acompañada de un programa de formación
  profesional. Apostando por el largo plazo y el capital humano. Una
  idea que también también defiende Stiglitz (Stiglitz, 2020:247).
\item
  Asegurar una remuneración fija superior al salario mínimo de forma
  que, incluso se podría prescindir de éste.
\item
  Orientar las actividades a políticas de cuidado del medioambiente, las
  personas o la comunidad.
\end{itemize}

Mazzucato propone que sea el Estado quien sufrague el programa, aunque
luego sean administraciones locales quienes administren. Evidentemente
la traslación de esta idea a cada país se debería hacer atendiendo a su
organización territorial y distribución de competencias. La idea que
parece defender es que la administración más cercana al individuo y la
realidad social es la más adecuada para gestionar el plan.

\hypertarget{dividendo-ciudadano}{%
\subsection{Dividendo ciudadano}\label{dividendo-ciudadano}}

El Dividendo ciudadano de Mazzucato se trataría de una reserva de la
riqueza que compartiría toda la población de forma colectiva. Así, los
ciudadanos percibirían una participación directa en lo que produce un
país. Casos como el petróleo de Alaska o una propuesta en California
sobre un dividendo por los beneficios obtenidos por los datos personales
serían dos ejemplos. Una real, otro no.

\hypertarget{feudalismo-digital}{%
\subsection{Feudalismo digital}\label{feudalismo-digital}}

En relación al uso y beneficio de los datos personales podemos hablar
del \emph{big data} y los algoritmos como un caso paradigmático de lo
que ha llegado a definir como Feudalismo digital. La gravedad de la
situación se da cuando corporaciones como Google, Amazon, etc. ya no se
dedican a proveernos de nuevos bienes que pueden interesarnos, sino que
ya saben más de nosotros que nosotros mismos. Además, que en sus
análisis y propuestas perpetúan los roles de género o raza.

La propuesta de Mazzucato es usar, entre otras, toda la información de
estas empresas para mejorar los servicios públicos que en estos momentos
están, contrariamente, erosionando.

De hecho, el desarrollo de software de rastreo para hacer frente al
Covid-19 fue muy dispar dependiendo de si se habían hecho inversiones
públicas anteriormente para hacer frente a crisis y emergencias
sanitarias.

\hypertarget{cadena-de-suministro}{%
\subsection{Cadena de suministro}\label{cadena-de-suministro}}

Debemos entender que sin la previa inversión pública en tecnología
(satélites, líneas de telefonía...) no se hubiesen podido desarrollar
muchas grandes corporaciones. Google, la pantalla táctil, el GPS, el
reconocimiento de voz o el Tesla no existirían sin una inversión pública
inicial en investigación.

Este argumento, la cadena de suministro, ha sido objeto de críticas,
como apuntaremos más adelante, por la economista liberal McCloskey.
Irónicamente, el también liberal Guy Sorman cita como ejemplo de
políticas acertadas en Europa, la desregularización de algunos sectores
por parte de la Comisión Europea que coinciden con los que Mazzucato
señala como beneficiados por la ``cadena de suministro''.

\hypertarget{asociaciuxf3n-puxfablico-privada.-interuxe9s-general}{%
\subsection{Asociación público-privada. Interés
general}\label{asociaciuxf3n-puxfablico-privada.-interuxe9s-general}}

Su reivindicación de convertir los debilitados Estados en ``Estados
emprendedores'' que inviertan en innovación, en sanidad pública y
energías renovables, pero sobre todo que aprendan a negociar con las
empresas privadas para que cualquier inversión de dinero público que
reciban acabe revirtiendo en la sociedad.

Mazzucato hace un repaso de la inversión pública que, mediante
asociación público-privada o subvenciones, se ha hecho en los últimos
años para investigar centrándose en las vacunas. Denuncia la ausencia de
una dirección común, unas reglas basadas en objetivos de interés público
por una parte y, exige compartir los resultados de las investigaciones
de las corporaciones frente a la competencia cerrada actual.

Es aquí donde la catedrática defendía que sean las autoridades
nacionales, el Estado, las que dirijan toda la investigación además de
asegurar un acceso universal de vacunas como la de Covid-19 evitando que
vuelva a repetirse lo ocurrido con la crisis del SARS en 2003 que pese a
la inversión de más de 700 millones de dólares por el gobierno de EEUU,
la vacuna tiene un precio demasiado alto para gran parte de la
población.

Por lo tanto, estamos ante una reivindicación de un doble papel del
Estado a la hora de negociar. Por una parte, que los beneficios
económicos repercutan en la sociedad, por otra, que la sea el interés
general quien oriente las investigaciones. Lo que encuadraríamos en dos
de las tres funciones generales que hemos atribuido al Estado.

\hypertarget{poluxedtica-verde}{%
\subsection{Política verde}\label{poluxedtica-verde}}

Mazzucato es creadora del Instituto para la Innovación y el Propósito
Público (IIPP) en el University College de Londres, entidad que trabaja
para que los gobiernos salgan de rincón en que se ocultaron en la década
de 1980 y actúen no solo reaccionando ante las crisis sino tomando las
riendas de una nueva relación entre empresa privada y sector público
para conseguir, como mínimo, los objetivos marcados en el Acuerdo de
París. El mercado de por sí no va a encontrar una solución verde, la
innovación debe ir de la mano de la regulación y el papel director del
Estado.

Así, de hecho, podemos constatar que la iniciativa europea por la
Economía Circular, en principio, no solo se ha transformado en un
paquete de ayudas para empresas. sino que se ha plasmado también en
forma de leyes nacionales\textbf{.}\footnote{1 En 2022 se publicó la
  Estrategia Española para la Economía Circular, Planes de Acción y la
  Ley 7/2022, de 8 de abril, de residuos y suelos contaminados para una
  economía circular.
  \url{https://www.miteco.gob.es/es/calidad-y-evaluacion-ambiental/temas/economia-circular/estrategia}}

\hypertarget{la-desigualdad.-hogares-en-la-brecha-digital}{%
\subsection{La desigualdad. Hogares en la brecha
digital}\label{la-desigualdad.-hogares-en-la-brecha-digital}}

Otra consecuencia del abandono de los Estados en su función de asegurar
que el progreso, sobre todo de los más desfavorecidos, se ha visto en la
pandemia cuando en tantos hogares el derecho a la educación lejos de ser
una realidad se ha quedado en papel mojado; muchos escolares no han
podido seguir las clases on-line. La brecha digital en los sectores
sociales más empobrecidos de países ricos nos obliga a repensar sobre la
necesidad de políticas públicas que atiendan a los más desfavorecidos
garantizándoles, al menos, el derecho a algo tan básico como la
educación.

La economista defiende el apoyo del Estado a las familias en la
condonación de las deudas incluso que aporte ayudas a las más
necesitadas.

\hypertarget{redistribuciuxf3n.-nueva-poluxedtica-fiscal-y-subsidio}{%
\subsection{Redistribución. Nueva política fiscal y
subsidio}\label{redistribuciuxf3n.-nueva-poluxedtica-fiscal-y-subsidio}}

Al igual que Piketty, Mazzucato reclama la necesidad de financiar
debidamente al Estado. Deshacer la senda de la revolución
\emph{reaganiana} o \emph{tatcheriana} y asumir que hace falta un sector
público que nos auxilie como ha demostrado esta crisis, inicialmente,
sanitaria. Una vez más una economista nos recuerda a Rooselvet, Keynes y
el New Deal vuelven a ser la lección de la historia que no deberíamos
olvidar.

La idea defendida por la profesora consiste en reformar el concepto de
subsidio involucrando a les empresas y a los mismos trabajadores además
de la administración.

En su idea de Gran Pacto para una Constitución Fiscal nueva pretende, de
forma análoga a lo que apuntaba Keynes, escapar de las viejas políticas
y iniciar una política fiscal donde se prime tanto ``el ritmo como la
dirección''. Es decir, erradicar la especulación y que esté presente
siempre el largo plazo, la estabilidad de empleos y negocios, pero
también en el problema grave ambiental.

\hypertarget{conclusiuxf3n-1}{%
\subsection{Conclusión}\label{conclusiuxf3n-1}}

La crisis nos trae la necesidad de un Estado capaz de ayudar a la gente,
a las empresas y de reforzar los servicios sanitarios; revitalizar la
inversión, la innovación y cambiar la manera de llevar a cabo la
colaboración público-privada. No se trata de más Estado o menos, sino de
mejor Estado, de un Estado diferente.

El sector público no solo debe corregir los fallos del mercado, sino que
también debe formar el mercado teniendo siempre presente la perspectiva
del largo plazo. Esta propuesta para salvar el capitalismo, evitar
nuevas crisis y, en todo caso, poder afrontarlas necesita de un Estado
que recupere plenas competencias y salga del replegamiento al que ha
sido condenado por los políticos occidentales desde las últimas décadas
del siglo pasado para empezar a orientar la actividad económica con
grandes objetivos a cumplir, las ``misiones''.

\hypertarget{mccloskey-contra-el-estatismo}{%
\section{5. McCLOSKEY, CONTRA EL
ESTATISMO}\label{mccloskey-contra-el-estatismo}}

Si hablamos del intervencionismo estatal en pro de la regulación o la
distribución y buscamos economistas actuales reacios no podemos pasar
por alto a Deirdre McCloskey. Esta economista e historiadora se opone a
toda forma de ``mangoneo'' del Estado: en esos términos.

La oposición al papel del Estado de la profesora de la Universidad de
Chicago no solamente abarca el ámbito de la economía como puedan ser
políticas fiscales, especialmente si tienen como finalidad la
redistribución; las regulaciones de negocios y profesiones, así como las
protecciones. McCloskey se opone a toda forma de control público o
estatismo, venga de donde venga. No solo en la economía enarbola su
bandera personal de la libertad¹.

\textbf{El Estado no puede planificar la economía}

Centrándonos en lo que serían las políticas económicas, McCloskey
entiende que a medida que las sociedades van haciéndose más complejas,
planificar su economía se convierte en algo peor que una quimera, algo
imposible. Esta idea de la economista se resume en su frase: ``Confiar
en el mercado es apostar por el esfuerzo de todas las mentes creativas''
(McCloskey, 2022) es el argumento que usa para defender el
\emph{laissez-faire}, la idea de \emph{que se encargue el mercado} que
tanto critican por simplista la izquierda según ella.

La destrucción creativa y la innovación son lo único que hay que
proteger, dicho lo cual se entiende que bien pocas reglas hacen falta
más que aquellas mínimas que aseguren el cumplimiento de contratos y
poco más.

\hypertarget{contra-la-protecciuxf3n}{%
\subsection{Contra la protección}\label{contra-la-protecciuxf3n}}

Se trata de una forma de intervencionismo que dificulta la deseada
competencia y, en todo caso, solo ayuda a los monopolios de empresas y
profesionales, también enemigos de esa sana competividad empresarial.

Las medidas proteccionistas como son los aranceles que la
derecha\footnote{1 Los liberales para Deirdre McCloskey no son la
  derecha ni a izquierda} y la izquierda quieren imponer acomodan al
inversor, desincentivando la innovación que es el germen del capitalismo
y favorecen al monopolio de empresas y profesionales.

\hypertarget{contra-la-regulaciuxf3n-del-estado}{%
\subsection{Contra la regulación del
Estado}\label{contra-la-regulaciuxf3n-del-estado}}

Guy Sorman explica en su comparativa entre la economía liberal americana
y la intervencionista europea continental que ``el papel del poder
público en EEUU no es proteger a las empresas existentes, sino facilitar
el ingreso al mercado de las innovadoras'' (Sorman, 2008) y compara el
número de tramites, días de retraso e importe inicial necesarios para
crear una empresa nueva en EEUU y en Francia. Las diferencias en el
momento de escribir su libro eran notables.

Para McCloskey, el exceso de regulación y de intervención administrativa
favorece la captura de rentas. Los liberales no han sido capaces de
introducir esta idea en la sociedad: cuanto mayor y más complejas sean
estas rentas capturables, mayor será la competencia por capturarlas,
mayores serán los recursos malbaratados en su captura y, por lo tanto,
mayor será el daño para el bienestar social. Las sobreregulación es el
terreno donde mejor se mueven los monopolios. Tanto los gremiales como
los de sectores estratégicos de gran importancia como energía o
comunicaciones.

\hypertarget{la-falacia-de-la-cadena-de-suministro}{%
\subsection{La falacia de la cadena de
suministro}\label{la-falacia-de-la-cadena-de-suministro}}

Entendiendo por falacia del suministro la idea de que las inversiones
públicas iniciales, son la condición necesaria para el desarrollo urbano
y que, además, estas inversiones estatales en grandes infraestructuras o
investigaciones son necesarias para el progreso.

McCloskey se pregunta por qué habiendo existido anteriormente
planificación en países como China el desarrollo no ocurría y afirma que
la condición o causa es haber apostado por la iniciativa privada, dando
libertad al promotor para edificar. Por eso se da a partir de 1991 en
China. Es este una de sus críticas y punto de desacuerdo como habremos
advertido con Mazucato quien precisamente usa este argumento para
amparar moralmente su propuesta de \emph{dividendo ciudadano.}

La economista liberal va más allá o, mejor dicho, tiene una visión
diametralmente opuesta. Las inversiones estatales no deberían ni
existir. Si cualquier inversión en infraestructuras como los trenes de
alta velocidad es rentable, ya las hará el mercado y las costeará el
usuario.

\hypertarget{contra-la-redistribuciuxf3n-de-estado}{%
\subsection{Contra la redistribución de
Estado}\label{contra-la-redistribuciuxf3n-de-estado}}

Para McCloskey, la redistribución es de un solo uso, puesto que se
consigue con una política fiscal fuerte y esto provoca un ``efecto
huida'' de los inversores hacia otro país. Y hay países que se han
especializado en recibirlos.

Piketty, como ya hemos visto, también presta atención a este problema.
Se trata de la competencia fiscal y es uno de los argumentos en favor de
su propuesta de crear impuesto sobre el capital, precisamente, mundial.

Es evidente que ambos economistas alertan sobre el problema de la
libertad de capitales o inversiones en una economía globalizada pero,
desde luego, no coinciden en la solución que cada cual da al problema.
Aunque más bien podríamos decir que Piketty propone una solución que,
reconoce difícil de llevar a la práctica, aunque defiende la viabilidad
de su implantación por fases mientras que McCloskey ``resuelve'' el
problema eliminándolo: si las empresas huyen de los impuestos, los
quitamos.

Sobre la redistribución, McCloskey reflexiona sobre un aspecto ético que
chocaría con el individualismo. A nadie le gusta llegar a graduarse,
conseguir un buen trabajo, seguir esforzándose por encontrar un empleo
más remunerado para acabar contribuyendo al subsidio de los que quizás
no se esfuerzan. Sin duda que en el trasfondo subyace esa visión
calvinista americana que tan bien describe Guy Sorman ``En EEUU se
sospecha que una persona pobre no trabaja lo suficiente para salir de
esa situación mientras que en esta Europa católica o marxista''una
persona pobre es percibida como una víctima del trabajo (Sorman,
2008:301).

En definitiva, la distribución es un problema o error desde el punto de
vista económico o práctico y, también desde un punto de vista ético para
McCloskey, aunque resulta mucho más interesante la comparación cultural
del franco-estadounidense Sorman.

El mismo Sorman, cuando repasa estas diferencias a ambos lados del
continente, cuestiona la utilidad de la redistribución que llevan a cabo
los Estados europeos al recordar que, pese a las ayudas públicas a los
pobres, la pobreza no haya desaparecido todavía en Europa.

La acusación de la historiadora y economista en su doble condición ha
lanzado va más lejos. Según ella, derecha e izquierda tienen claro que
en la distribución siempre se debe beneficiar a las clases medias que
dominan las urnas. Con lo cual, la pretendida justicia social que
Bismark expropió a la socialdemocracia queda en entredicho. Y la
preocupación por los parias, dudosa. Todas estas ayudas acaban siendo
una suerte de proteccionismo clientelar (subsidios agrícolas,
aranceles\ldots)

\hypertarget{contra-todo-proteccionismo}{%
\subsection{Contra todo
proteccionismo}\label{contra-todo-proteccionismo}}

Hemos introducido el proteccionismo del Estado con las empresas locales,
pero McCloskey critica todo proteccionismo de izquierdas y derechas,
afecte a empresas, profesionales libres o trabajadores. El
proteccionismo para ella es populismo o nacionalismo, por perjudicar la
competencia libre que es la única vía de obligar a que las empresa y
profesionales de cada país mejoren en un mundo (mercado empresarial,
profesional y laboral) sin fronteras. Sin fronteras para industriales,
médicos o jornaleros.

A esta idea Guy Sorman en su crítica contra el proteccionismo europeo
llega a afirmar que, si los gobiernos europeos dejasen funcionar a los
mercados, desaparecerían un mayor número de empresas, pero podrían
crearse muchas más'' recuperándose el empleo incluso la calidad de éste
( Sorman, 2021:304).

En su libro, Sorman se hace eco del fracaso de propuestas de economistas
para acabar con los sistemas de subvenciones comprando las actividades
(agrícolas principalmente) con el dinero que esperan recibir de
subvenciones en 10 años, por ejemplo; innovar, vender unas y hacer más
viables las otras posteriormente... También expone otras propuestas para
la compra de licencias como el taxi para negociar posteriormente
consiguiendo tarifas más baratas etc.

\hypertarget{el-desempleo-y-la-libertad-de-mercado}{%
\subsection{El desempleo y la libertad de
mercado}\label{el-desempleo-y-la-libertad-de-mercado}}

Para Mc Closkey una de las medidas que deberían tomar los gobiernos para
reducir el desempleo, sobre todo el paro juvenil, es eliminar el salario
mínimo. Éste actúa como barrera impidiendo que se pueda, aunque sea
inicialmente, contratar a jóvenes si experiencia. No se hace eco en este
punto de las políticas que en Europa se llevan a cabo regulando
mendiante becas o contratos de aprendizaje aunque se puede entender que
también son una forma de intervencionismo y subvención pública
indeseable desde la óptica suya.

El paro según McCloskey se resuelve con la adaptación a nuevas
necesidades, ni exprimiendo al ``capitalista''. Respecto a lo que
denomina el desempleo tecnológico, recuerda que el fin de la economía es
producir, no conservar empleos. La innovación como ya se ha dicho, se
basa en la ``destrucción creativa'', por lo tanto, este ``desempleo''
siempre ha sido inevitable. El Estado no debe proteger al trabajador con
subsidios. Además, el trabajador debe reciclarse por su cuenta, él es
quien sabe qué quiere y necesita. Los subsidios masivos son una
injusticia para el resto de los trabajadores y una ruina para el
conjunto.

Esto nos recuerda a una de las patas del Programa de Empleo Público de
Mazzucato: la formación profesional para el personal desempleado
pensando en el largo plazo de la economía. También la formación como
condicionante o estímulo a las empresas receptoras de ayudas. Es
evidente que, en ambas economistas, McCloskey y Mazzucato, coinciden en
ver la necesidad de mejorar las capacitaciones del desempleado.
Discrepan evidentemente, en lo de siempre: en quien debe costearlo. Para
Mazzucato debe ser el Estado y la empresa, para McCloskey, el
desempleado debe buscarse la vida porqué él sabe mejor que nadie qué
necesita. Podría objetarse que seguramente los más excluidos
probablemente ni lo sepan ni tengan recursos ya.

Stiglitz por contra considera que los salarios como los precios no
reaccionan a los cambios del mercado tan rápidamente como argumentan los
economistas liberales y, precisamente, por ello defiende una política de
estabilización del gobierno además de los bancos centrales en busca de
una eficiencia mayor de acuerdo con Pareto.

\hypertarget{conclusiuxf3n-2}{%
\subsection{Conclusión}\label{conclusiuxf3n-2}}

Las ideas liberales de Deirdre McCloskey y otros como Guy Sorman son
claramente contrarias a cualquier papel del Estado en la economía que no
sea el de garantizar un marco legal mínimo para que se cumplan los
acuerdos comerciales y se respete la propiedad privada.

La planificación, las restricciones, los aranceles y prohibiciones desde
los despachos ministeriales sean socialdemócratas o de derechas se
venden bien electoralmente por ser medidas populistas, pero a medio y
largo plazo tienen efectos nocivos sobre la economía.

Las ayudas a empresas no son una solución, al contrario, un problema.
Las ayudas sociales solo son una forma de comprar los votos de la clase
media. Las protecciones a sectores son un atentado a la libre
competencia que tiene por objetivo bajar los precios de forma inmediata
e indiscutible.

\hypertarget{bibliografuxeda}{%
\section{6. BIBLIOGRAFÍA}\label{bibliografuxeda}}

\begin{enumerate}
\def\labelenumi{\arabic{enumi}.}
\item
  Mazzucato, Mariana 2021, ``No desaprovechemos esta crisis''. Galaxia
  Gutenberg.
\item
  McCloskey, Deirdre 2019 ``¿Por qué el liberalismo funciona?''. Deusto.
\item
  Piketty, Thomas 2014, ``El capital en el siglo XXI''. Madrid, Fondo de
  Cultura Económica.
\item
  Sorman, Guy 2008, ``La economía no miente'', Madrid, Ed. Fundación
  FAES, SL
\item
  Stiglitz, Joseph E. 2020, ``El capitalismo progresista''. Penguin
  Random House Grupo Editorial.
\item
  Vergés, Joaquim. ``Balance de las políticas de privatización de
  empresas públicas en España (1985-1999)''. Ministerio de Industria,
  Comercio y Consumo, 1999
  \href{https://www.mincotur.gob.es/Publicaciones/Publicacionesperiodicas/EconomiaIndustrial/RevistaEconomiaIndustrial/330/17jve.pdf}{\ul{https://www.mincotur.gob.es/Publicaciones/Publicacionesperiodicas/EconomiaIndustrial/RevistaEconomiaIndustrial/330/17jve.pdf}}
\end{enumerate}

\end{document}
